

%🍁% \chapter{Case study: data structure selection  |  案例研究:数据结构选择}
\chapter{案例研究:数据结构选择}

%🍁% At this point you have learned about Python's core data structures,
%🍁% and you have seen some of the algorithms that use them.
%🍁% If you would like to know more about algorithms, this might be a good
%🍁% time to read Chapter~\ref{algorithms}.
%🍁% But you don't have to read it before you go on; you can read
%🍁% it whenever you are interested.

目前为止, 你已经学完了 Python 的核心数据结构, 同时你也接触了利用到这些数据结构的一些算法。
如果你希望学习更多算法知识, 那么现在是阅读 附录~\ref{algorithms} 的好时机。
但是不必急着马上读, 什么时候感兴趣了再去读即可。

%🍁% This chapter presents a case study with exercises that let
%🍁% you think about choosing data structures and practice using them.

本章是一个案例研究, 同时给出了一些习题, 目的是启发你思考如何选择数据结构, 并练习数据结构使用。

%🍁% \section{Word frequency analysis  |  词频分析}
\section{词频分析}
\label{analysis}

%🍁% As usual, you should at least attempt the exercises
%🍁% before you read my solutions.

和之前一样, 在查看答案之前, 你可以试着解答一下这些习题。

\begin{exercise}

%🍁% Write a program that reads a file, breaks each line into
%🍁% words, strips whitespace and punctuation from the words, and
%🍁% converts them to lowercase.

编写一个程序, 读取一个文件, 将每一行转换成单词列表, 删掉单词中的空格和标点, 然后将它们转换为小写字母。

\index{string module}
\index{module!string}

%🍁% Hint: The {\tt string} module provides a string named {\tt whitespace},
%🍁% which contains space, tab, newline, etc., and {\tt
%🍁%   punctuation} which contains the punctuation characters.  Let's see
%🍁% if we can make Python swear:

提示: {\em \li{string}} 模块提供了名为 {\em \li{whitespace}} 的字符串,
其包括空格、制表符、新行等等, 以及名为 {\em \li{punctuation}} 的字符串,
其包括标点字符。   试试能否让 {\em Python} 说脏话:

\begin{em}
\begin{lstlisting}
>>> import string
>>> string.punctuation
'!"#$%&'()*+,-./:;<=>?@[\]^_`{|}~'
\end{lstlisting}
\end{em}

%🍁% %
%🍁% Also, you might consider using the string methods {\tt strip},
%🍁% {\tt replace} and {\tt translate}.

同时, 你可以考虑使用字符串方法 {\em \li{strip}} 、 {\em \li{replace}} 和 {\em \li{translate}}。

\index{strip method}
\index{method!strip}
\index{replace method}
\index{method!replace}
\index{translate method}
\index{method!translate}
\end{exercise}

\begin{exercise}
\index{Project Gutenberg}

%🍁% Go to Project Gutenberg (\url{http://gutenberg.org}) and download
%🍁% your favorite out-of-copyright book in plain text format.

前往\href{http://gutenberg.org}{古登堡项目\footnote{Project Gutenberg, \url{http://gutenberg.org}.}}, 以纯文本格式下载你喜欢的已无版权保护的图书。

\index{plain text}
\index{text!plain}

%🍁% Modify your program from the previous exercise to read the book
%🍁% you downloaded, skip over the header information at the beginning
%🍁% of the file, and process the rest of the words as before.
%🍁% Then modify the program to count the total number of words in
%🍁% the book, and the number of times each word is used.

修改前面习题的程序, 读取你下载的书,
跳过文件开始的头部信息, 像之前那样处理其余的单词。
然后修改程序, 计算书中单词的总数, 以及每个单词使用的次数。

\index{word frequency}
\index{frequency!word}

%🍁% Print the number of different words used in the book.  Compare
%🍁% different books by different authors, written in different eras.
%🍁% Which author uses the most extensive vocabulary?

打印该书使用单词的总数。   比较不同年代、不同作者写的书。
哪个作者使用的词汇量最大?

\end{exercise}

\begin{exercise}
%🍁% Modify the program from the previous exercise to print the
%🍁% 20 most frequently used words in the book.

修改上一个习题中的程序, 打印书中最常使用的 {\em 20} 个单词。
\end{exercise}

\begin{exercise}
%🍁% Modify the previous program to read a word list (see
%🍁% Section~\ref{wordlist}) and then print all the words in the book that
%🍁% are not in the word list.  How many of them are typos?  How many of
%🍁% them are common words that {\em should} be in the word list, and how
%🍁% many of them are really obscure?

修改上一个习题中的程序, 读取一个单词列表(见 {\em \ref{wordlist}}~节),
然后打印书中所有没有出现在该单词表中的单词。
它们中有多少是拼写错误的?有多少是词表中 {\bf 应该} 包括的常用词?有多少是生僻词?
\end{exercise}

%🍁% \section{Random numbers  |  随机数}
\section{随机数}
\index{random number}
\index{number, random}
\index{deterministic}
\index{pseudorandom}

%🍁% Given the same inputs, most computer programs generate the same
%🍁% outputs every time, so they are said to be {\bf deterministic}.
%🍁% Determinism is usually a good thing, since we expect the same
%🍁% calculation to yield the same result.  For some applications, though,
%🍁% we want the computer to be unpredictable.  Games are an obvious
%🍁% example, but there are more.

给定相同的输入, 大多数计算机程序每次都会生成相同的输出,
它们因此被称作 {\bf 确定性的\footnote{deterministic}}  。
确定性通常是个好东西, 因为我们期望相同的计算产生相同的结果。
然而, 对于有些应用, 我们希望计算机不可预知。
游戏是一个明显的例子, 但是不限于此。

%🍁% Making a program truly nondeterministic turns out to be difficult,
%🍁% but there are ways to make it at least seem nondeterministic.  One of
%🍁% them is to use algorithms that generate {\bf pseudorandom} numbers.
%🍁% Pseudorandom numbers are not truly random because they are generated
%🍁% by a deterministic computation, but just by looking at the numbers it
%🍁% is all but impossible to distinguish them from random.

让程序具备真正意义上的非确定性并不容易, 但是有办法使它至少看起来是不确定的。
其中之一是使用生成 {\em 伪随机数} (pseudorandom) 的算法。
伪随机数不是真正的随机数, 因为它们由一个确定性的计算生成,
但是仅看其生成的数字, 不可能将它们和随机生成的相区分开。

\index{random module}
\index{module!random}

%🍁% The {\tt random} module provides functions that generate
%🍁% pseudorandom numbers (which I will simply call ``random'' from
%🍁% here on).

\li{random} 模块提供了生成伪随机数(下文中简称为``随机数'')的函数。

\index{random function}
\index{function!random}

%🍁% The function {\tt random} returns a random float
%🍁% between 0.0 and 1.0 (including 0.0 but not 1.0).  Each time you
%🍁% call {\tt random}, you get the next number in a long series.  To see a
%🍁% sample, run this loop:

函数 \li{random} 返回一个 0.0 到 1.0 之间的随机浮点数(包括 0.0 , 但是不包括 1.0 )。    每次调用 \li{random} , 你获得一个长序列中的下一个数。
举个例子, 运行此循环:

\begin{lstlisting}
import random
for i in range(10):
    x = random.random()
    print(x)
\end{lstlisting}

%🍁% %
%🍁% The function {\tt randint} takes parameters {\tt low} and
%🍁% {\tt high} and returns an integer between {\tt low} and
%🍁% {\tt high} (including both).

函数 \li{randint} 接受参数 \li{low} 和 \li{high} ,
返回一个 \li{low} 和 \li{high} 之间的整数(两个都包括)。

\index{randint function}
\index{function!randint}

\begin{lstlisting}
>>> random.randint(5, 10)
5
>>> random.randint(5, 10)
9
\end{lstlisting}

%🍁% %
%🍁% To choose an element from a sequence at random, you can use
%🍁% {\tt choice}:

你可以使用 \li{choice} , 从一个序列中随机选择一个元素:

\index{choice function}
\index{function!choice}

\begin{lstlisting}
>>> t = [1, 2, 3]
>>> random.choice(t)
2
>>> random.choice(t)
3
\end{lstlisting}

%🍁% %
%🍁% The {\tt random} module also provides functions to generate
%🍁% random values from continuous distributions including
%🍁% Gaussian, exponential, gamma, and a few more.

\li{random} 模块提供的函数, 还可以生成符合高斯、指数、伽马等连续分布的随机值。

\begin{exercise}
\index{histogram!random choice}

%🍁% Write a function named \verb"choose_from_hist" that takes
%🍁% a histogram as defined in Section~\ref{histogram} and returns a
%🍁% random value from the histogram, chosen with probability
%🍁% in proportion to frequency.  For example, for this histogram:

编写一个名为 {\em \li{choose_from_hist}} 的函数,  其接受一个如 {\em \ref{histogram}} 一节中定义的 {\em \li{histogram}} 对象作为参数,  并从该对象中返回一个随机值,  其选择概率和值出现的频率成正比。   例如:

\begin{em}
\begin{lstlisting}
>>> t = ['a', 'a', 'b']
>>> hist = histogram(t)
>>> hist
{'a': 2, 'b': 1}
\end{lstlisting}
\end{em}

%🍁% %
%🍁% your function should return \verb"'a'" with probability $2/3$ and \verb"'b'"
%🍁% with probability $1/3$.

你的函数返回 {\em \li{'a'}} 的概率应该是 {\em $2/3$} , 返回 {\em \li{'b'}} 的概率应该是 {\em $1/3$} 。

\end{exercise}

%🍁% \section{Word histogram  |  单词直方图}
\section{单词直方图}

%🍁% You should attempt the previous exercises before you go on.
%🍁% You can download my solution from
%🍁%  \url{http://thinkpython2.com/code/analyze_book1.py}.  You will
%🍁% also need \url{http://thinkpython2.com/code/emma.txt}.
%🍁% Here is a program that reads a file and builds a histogram of the
%🍁% words in the file:

在继续下面的习题之前, 你应该尝试完成前面的练习。
你可以下载 \href{http://thinkpython2.com/code/analyze_book1.py}{我的答案}。
你还需要下载\href{http://thinkpython2.com/code/emma.txt}{这个文件} 。
下面这个程序将读取一个文件, 并建立文件中单词的直方图:

\index{histogram!word frequencies}

\begin{lstlisting}
import string
def process_file(filename):
    hist = dict()
    fp = open(filename)
    for line in fp:
        process_line(line, hist)
    return hist
def process_line(line, hist):
    line = line.replace('-', ' ')
    for word in line.split():
        word = word.strip(string.punctuation + string.whitespace)
        word = word.lower()
        hist[word] = hist.get(word, 0) + 1
hist = process_file('emma.txt')
\end{lstlisting}

%🍁% %
%🍁% This program reads {\tt emma.txt}, which contains the text of {\em
%🍁%   Emma} by Jane Austen.

该程序读取 \li{emma.txt} , 其包括 Jane Austen 的小说 《{\em Emma}》 文本。

\index{Austin, Jane}

%🍁% \verb"process_file" loops through the lines of the file,
%🍁% passing them one at a time to \verb"process_line".  The histogram
%🍁% {\tt hist} is being used as an accumulator.

\li{process_file} 循环读取文件的每行, 依次把它们传递给 \li{process_line} 。
直方图 \li{hist} 被用作一个累加器。

\index{accumulator!histogram}
\index{traversal}

%🍁% \verb"process_line" uses the string method {\tt replace} to replace
%🍁% hyphens with spaces before using {\tt split} to break the line into a
%🍁% list of strings.  It traverses the list of words and uses {\tt strip}
%🍁% and {\tt lower} to remove punctuation and convert to lower case.  (It
%🍁% is a shorthand to say that strings are ``converted''; remember that
%🍁% strings are immutable, so methods like {\tt strip} and {\tt lower}
%🍁% return new strings.)

\li{process_line} 使用字符串的 \li{replace} 方法将连字符替换成空格。
它会遍历单词列表, 并使用 \li{strip} 和 \li{lower} 来删除标点以及将单词转换为小写。
(``转换'' 只是一种简略的说法;记住, 字符串是不可变的, 所以类似 \li{strip} 和 \li{lower} 这样的方法其实返回的是新字符串。  )

%🍁% Finally, \verb"process_line" updates the histogram by creating a new
%🍁% item or incrementing an existing one.

最后, \li{process_line} 通过生成一个新的项或者递增一个已有的项来更新直方

\index{update!histogram}

%🍁% To count the total number of words in the file, we can add up
%🍁% the frequencies in the histogram:

我们可以通过累加直方图中的频率, 来统计文件中的单词总数:

\begin{lstlisting}
def total_words(hist):
    return sum(hist.values())
\end{lstlisting}

%
%🍁% The number of different words is just the number of items in
%🍁% the dictionary:

不同单词的数量恰好是词典中项的数目:

\begin{lstlisting}
def different_words(hist):
    return len(hist)
\end{lstlisting}

%🍁% %
%🍁% Here is some code to print the results:

以下打印结果的代码:

\begin{lstlisting}
print('Total number of words:', total_words(hist))
print('Number of different words:', different_words(hist))
\end{lstlisting}

%🍁% %
%🍁% And the results:

其结果是:

\begin{lstlisting}
Total number of words: 161080
Number of different words: 7214
\end{lstlisting}

%
%🍁% \section{Most common words  |  最常用单词}
\section{最常用单词}

%🍁% To find the most common words, we can make a list of tuples,
%🍁% where each tuple contains a word and its frequency,
%🍁% and sort it.
%🍁% The following function takes a histogram and returns a list of
%🍁% word-frequency tuples:

为了找到最常用的单词, 我们可以使用元组列表, 其中每个元组包含单词和它的频率, 然后排序这个列表。

下面的函数接受一个直方图并且返回一个
单词-频率的元组列表:

\begin{lstlisting}
def most_common(hist):
    t = []
    for key, value in hist.items():
        t.append((value, key))
    t.sort(reverse=True)
    return t
\end{lstlisting}

%🍁% In each tuple, the frequency appears first, so the resulting list is
%🍁% sorted by frequency.  Here is a loop that prints the ten most common
%🍁% words:

每一个元组中, 频率在前, 所以这个列表是按照频率排序。
下面是输出最常用的十个单词的循环:

\begin{lstlisting}
t = most_common(hist)
print('The most common words are:')
for freq, word in t[:10]:
    print(word, freq, sep='\t')
\end{lstlisting}

%🍁% %
%🍁% I use the keyword argument {\tt sep} to tell {\tt print} to use a tab
%🍁% character as a ``separator'', rather than a space, so the second
%🍁% column is lined up.  Here are the results from {\em Emma}:

这里我通过关键词参数 \li{sep},  让 \li{print} 使用一个制表符(Tab)而不是空格键作为分隔符,  所以第二行将对齐。    下面是对 《{\em Emma}》 的分析结果:

\begin{lstlisting}
The most common words are:
to      5242
the     5205
and     4897
of      4295
i       3191
a       3130
it      2529
her     2483
was     2400
she     2364
\end{lstlisting}

%🍁% %
%🍁% This code can be simplified using the {\tt key} parameter of
%🍁% the {\tt sort} function.  If you are curious, you can read about it
%🍁% at \url{https://wiki.python.org/moin/HowTo/Sorting}.

当然, 这段代码也可以通过 \li{sort} 函数的参数 \li{key} 进行简化。
如果你感兴趣, 可以阅读\href{https://wiki.python.org/moin/HowTo/Sorting}{这篇文章} 。

%🍁% \section{Optional parameters  |  可选形参}
\section{可选形参}

\index{optional parameter}
\index{parameter!optional}

%🍁% We have seen built-in functions and methods that take optional
%🍁% arguments.  It is possible to write programmer-defined functions
%🍁% with optional arguments, too.  For example, here is a function that
%🍁% prints the most common words in a histogram

我们已经见过接受可变数量实参的函数和方法了。
程序员也可以自己定义具有可选实参的函数。
例如, 下面就是一个打印直方图中最常见单词的函数。

\index{programmer-defined function}
\index{function!programmer defined}

\begin{lstlisting}
def print_most_common(hist, num=10):
    t = most_common(hist)
    print('The most common words are:')
    for freq, word in t[:num]:
        print(word, freq, sep='\t')
\end{lstlisting}

%🍁% The first parameter is required; the second is optional.
%🍁% The {\bf default value} of {\tt num} is 10.

第一个形参是必须的;第二个是可选的。   \li{num} 的\ **默认值(default
value)**\ 是10。

\index{default value}
\index{value!default}

%🍁% If you only provide one argument:

如果你只提供了一个实参:

\begin{lstlisting}
print_most_common(hist)
\end{lstlisting}
{\tt num} gets the default value.  If you provide two arguments:
\begin{lstlisting}
print_most_common(hist, 20)
\end{lstlisting}
{\tt num} gets the value of the argument instead.  In other
words, the optional argument {\bf overrides} the default value.

\index{override}

%🍁% If a function has both required and optional parameters, all
%🍁% the required parameters have to come first, followed by the
%🍁% optional ones.

如果一个函数同时有必选和可选两类形参, 则所有的必选形参必须首先出现, 可选形参紧随其后。

%🍁% \section{Dictionary subtraction  |  字典差集}
\section{字典差集}

\label{dictsub}
\index{dictionary!subtraction}
\index{subtraction!dictionary}

%🍁% Finding the words from the book that are not in the word list
%🍁% from {\tt words.txt} is a problem you might recognize as set
%🍁% subtraction; that is, we want to find all the words from one
%🍁% set (the words in the book) that are not in the other (the
%🍁% words in the list).

从书中找到所有没出现在词表 \li{words.txt} 中的单词, 可以认为是一个差集问题;
也就是, 我们应该从一个集合中(书中的单词)找到所有没出现在另一个集合中
(列表中的单词)的单词。

%🍁% {\tt subtract} takes dictionaries {\tt d1} and {\tt d2} and returns a
%🍁% new dictionary that contains all the keys from {\tt d1} that are not
%🍁% in {\tt d2}.  Since we don't really care about the values, we
%🍁% set them all to None.

\li{subtract} 接受词典 \li{d1} 和 \li{d2} , 并返回一个新的词典,
其包括 \li{d1} 中的所有没出现在 \li{d2} 中的键。
由于并不真正关心值是什么, 我们将它们都设为 \li{None}。

\begin{lstlisting}
def subtract(d1, d2):
    res = dict()
    for key in d1:
        if key not in d2:
            res[key] = None
    return res
\end{lstlisting}

%🍁% %
%🍁% To find the words in the book that are not in {\tt words.txt},
%🍁% we can use \verb"process_file" to build a histogram for
%🍁% {\tt words.txt}, and then subtract:

为了找到书中没有出现在 \li{words.txt} 中的单词,
我们可以使用 \li{process_file} 来为 \li{words.txt} 构建一个直方图,
然后使用 \li{subtract} :

\begin{lstlisting}
words = process_file('words.txt')
diff = subtract(hist, words)
print("Words in the book that aren't in the word list:")
for word in diff.keys():
    print(word, end=' ')
\end{lstlisting}

%🍁% %
%🍁% Here are some of the results from {\em Emma}:

这是针对小说 《{\em Emma}》 的部分运行结果:

\begin{lstlisting}
Words in the book that aren't in the word list:
rencontre jane's blanche woodhouses disingenuousness
friend's venice apartment ...
\end{lstlisting}

%🍁% %
%🍁% Some of these words are names and possessives.  Others, like
%🍁% ``rencontre'', are no longer in common use.  But a few are common
%🍁% words that should really be in the list!

这些单词中, 一些是名字和名词所有格。  如“rencontre”这样的其他单词已经不常使用了。
但是有一些真的应该包括在列表中!

\begin{exercise}
\index{set}
\index{type!set}
%🍁% Python provides a data structure called {\tt set} that provides many
%🍁% common set operations.  You can read about them in Section~\ref{sets},
%🍁% or read the documentation at
%🍁% \url{http://docs.python.org/3/library/stdtypes.html#types-set}.
%🍁% Write a program that uses set subtraction to find words in the book
%🍁% that are not in the word list.  Solution:
%🍁% \url{http://thinkpython2.com/code/analyze_book2.py}.

Python 提供了一个叫做集合 (set) 的数据结构, 支持许多常见的集合操作。
你可以前往第十九章阅读相关内容, 或者阅读\href{http://docs.python.org/3/library/stdtypes.html#types-set}{官方文档} 。

编写一个程序, 使用集合的差集操作来找出一本书中不在 \li{work list} 中的单词。

\href{http://thinkpython2.com/code/analyze_book2.py}{参考答案}

\end{exercise}

%🍁% \section{Random words  |  随机单词}
\section{随机单词}
\label{randomwords}
\index{histogram!random choice}

%🍁% To choose a random word from the histogram, the simplest algorithm
%🍁% is to build a list with multiple copies of each word, according
%🍁% to the observed frequency, and then choose from the list:

如果想从直方图中随机选择一个单词, 最简单的算法是创建一个列表,
其中根据其出现的频率, 每个单词都有相应个数的拷贝, 然后从该列表中选择单词:

\begin{lstlisting}
def random_word(h):
    t = []
    for word, freq in h.items():
        t.extend([word] * freq)
    return random.choice(t)
\end{lstlisting}

%🍁% %
%🍁% The expression {\tt [word] * freq} creates a list with {\tt freq}
%🍁% copies of the string {\tt word}.  The {\tt extend}
%🍁% method is similar to {\tt append} except that the argument is
%🍁% a sequence.

表达式 \li{[word] * freq} 创建一个具有 \li{freq} 个 \li{word} 字符串拷贝的列表。
除了它的实参要求是一个序列外, \li{extend} 方法和 \li{append} 方法很像。

%🍁% This algorithm works, but it is not very efficient; each time you
%🍁% choose a random word, it rebuilds the list, which is as big as
%🍁% the original book.  An obvious improvement is to build the list
%🍁% once and then make multiple selections, but the list is still big.
%🍁% An alternative is:

该算法能够满足要求, 但是效率不够高;
每次你选择一个随机单词, 它都重建列表, 这个列表和原书一样大。
一个明显的改进是, 创建列表一次, 然后进行多次选择,  但是该列表仍然很大。

\begin{enumerate}
%🍁% \item Use {\tt keys} to get a list of the words in the book.

\item 使用 \li{keys} 来获得该书中单词的列表。

%🍁% \item Build a list that contains the cumulative sum of the word
%🍁%   frequencies (see Exercise~\ref{cumulative}).  The last item
%🍁%   in this list is the total number of words in the book, $n$.

\item 创建一个包含单词频率累积和的列表(见 练习~\ref{cumulative})。    此列表的最后一项是书中单词的数目 $n$ 。

%🍁% \item Choose a random number from 1 to $n$.  Use a bisection search
%🍁%   (See Exercise~\ref{bisection}) to find the index where the random
%🍁%   number would be inserted in the cumulative sum.

\item 选择一个从 1 到 $n$ 的随机数。   使用二分搜索(见\ref{bisection}节)  找到该随机数应该被在累积和中插入的索引。

%🍁% \item Use the index to find the corresponding word in the word list.

\item 使用该索引从单词列表中找到相应的单词。
\end{enumerate}

\begin{exercise}
\label{randhist}
\index{algorithm}
%🍁% Write a program that uses this algorithm to choose a random word from
%🍁% the book.  Solution:
%🍁% \url{http://thinkpython2.com/code/analyze_book3.py}.
\end{exercise}

编写一个使用该算法从书中选择一个随机单词的程序。
\href{http://thinkpython2.com/code/analyze_book3.py}{参考答案}


%🍁% \section{Markov analysis  |  马尔科夫分析}
\section{马尔科夫分析}
\label{markov}
\index{Markov analysis}

%🍁% If you choose words from the book at random, you can get a
%🍁% sense of the vocabulary, but you probably won't get a sentence:

如果你从书中随机选择单词, 那么你会大致了解其使用的词汇, 但可能不会得到一个完整的句子:

\begin{lstlisting}
this the small regard harriet which knightley's it most things
\end{lstlisting}

%🍁% %
%🍁% A series of random words seldom makes sense because there
%🍁% is no relationship between successive words.  For example, in
%🍁% a real sentence you would expect an article like ``the'' to
%🍁% be followed by an adjective or a noun, and probably not a verb
%🍁% or adverb.

一系列随机单词很少有意义, 因为相邻的单词之间没有关系。
例如, 在一个真实的句子中, 你可能期望 ``the'' 这样的冠词后面跟着的是一个形容词或者名词,
而大不可能会是一个动词或者副词。

%🍁% One way to measure these kinds of relationships is Markov
%🍁% analysis, which
%🍁% characterizes, for a given sequence of words, the probability of the
%🍁% words that might come next.  For example, the song {\em Eric, the Half a
%🍁%   Bee} begins:

衡量相邻单词关系的方法之一是马尔科夫分析法, 对于一个给定的单词序列,
马尔科夫分析法将给出接下来单词的概率。   例如, 歌曲 {\em Eric, the Half a
Bee} 的开头是:

\begin{quote}
Half a bee, philosophically, \\
Must, ipso facto, half not be. \\
But half the bee has got to be \\
Vis a vis, its entity. D'you see? \\
\\
But can a bee be said to be \\
Or not to be an entire bee \\
When half the bee is not a bee \\
Due to some ancient injury? \\
\end{quote}

%🍁% %
%🍁% In this text,
%🍁% the phrase ``half the'' is always followed by the word ``bee'',
%🍁% but the phrase ``the bee'' might be followed by either
%🍁% ``has'' or ``is''.

在此文本中, 短语 ``half the'' 后面总是跟着单词 ``bee'',  但是短语 ``the
bee'' 则可能跟着 ``has'' 或者  ``is''。

\index{prefix}
\index{suffix}
\index{mapping}

%🍁% The result of Markov analysis is a mapping from each prefix
%🍁% (like ``half the'' and ``the bee'') to all possible suffixes
%🍁% (like ``has'' and ``is'').

马尔科夫分析的结果是从每个前缀(如``half the''和``the bee'')
到所有可能的后缀(如``has'' 和 ``is'')的映射。

\index{random text}
\index{text!random}

%🍁% Given this mapping, you can generate a random text by
%🍁% starting with any prefix and choosing at random from the
%🍁% possible suffixes.  Next, you can combine the end of the
%🍁% prefix and the new suffix to form the next prefix, and repeat.
%🍁%
%🍁% For example, if you start with the prefix ``Half a'', then the
%🍁% next word has to be ``bee'', because the prefix only appears
%🍁% once in the text.  The next prefix is ``a bee'', so the
%🍁% next suffix might be ``philosophically'', ``be'' or ``due''.
%🍁%
%🍁% In this example the length of the prefix is always two, but
%🍁% you can do Markov analysis with any prefix length.

给定此映射, 你能够通过以任意前缀开始并从可能的后缀中随机选择一个的方法, 来生成一个随机文本。
接下来, 你可以将前缀的结尾和新的后缀组合成下一个前缀, 并重复下去。

例如, 如果你以前缀``Half a''开始, 然后下一个但是必须是``bee'',
因为此前缀在文本中仅出现一次。  下一个前缀是``a bee'',
所以下一个后缀可能是``philosophically'', ``be''或``due''。

此例中, 前缀的长度总是2, 但是你可以以任意前缀长度进行马尔科夫分析。
前缀的长度被称作此分析的“阶”。

\begin{exercise}
%🍁% Markov analysis:
马尔科夫分析:
\begin{enumerate}
%🍁% \item Write a program to read a text from a file and perform Markov
%🍁% analysis.  The result should be a dictionary that maps from
%🍁% prefixes to a collection of possible suffixes.  The collection
%🍁% might be a list, tuple, or dictionary; it is up to you to make
%🍁% an appropriate choice.  You can test your program with prefix
%🍁% length two, but you should write the program in a way that makes
%🍁% it easy to try other lengths.

\item 编写一个程序, 从一个文件中读取文本并执行马尔科夫分析。
   结果应该是一个字典, 即从前缀映射到一个可能的后缀集合。
   此后缀集合可以是一个列表、元组或字典;你需要做出合适的选择。
   你可以用长度为2的前缀测试程序, 但是在编写程序时, 应确保其很容易支持其它长度。

%🍁% \item Add a function to the previous program to generate random text
%🍁% based on the Markov analysis.  Here is an example from {\em Emma}
%🍁% with prefix length 2:
%🍁%
%🍁% \begin{quote}
%🍁% He was very clever, be it sweetness or be angry, ashamed or only
%🍁% amused, at such a stroke. She had never thought of Hannah till you
%🍁% were never meant for me?" "I cannot make speeches, Emma:" he soon cut
%🍁% it all himself.
%🍁% \end{quote}
%🍁%
%🍁% For this example, I left the punctuation attached to the words.
%🍁% The result is almost syntactically correct, but not quite.
%🍁% Semantically, it almost makes sense, but not quite.
%🍁%
%🍁% What happens if you increase the prefix length?  Does the random
%🍁% text make more sense?

\item 在前面的程序中添加一个函数, 基于马尔科夫分析生成随机文本。
   下面是使用 《{\em Emma} 》 执行前缀为2的马尔科夫分析生成的示例:

\begin{quote}
He was very clever, be it sweetness or be angry, ashamed or only
amused, at such a stroke. She had never thought of Hannah till you
were never meant for me?" "I cannot make speeches, Emma:" he soon cut
it all himself.
\end{quote}

在此例中, 我保留了附在词后面的标点符号。  从语法上看, 结果几乎是正确的, 但不完全。
语义上讲, 它几乎有意义, 但也不完全。

如果你增加前缀的长度, 会发生什么?随机文本更有意义是么?

%🍁% \item Once your program is working, you might want to try a mash-up:
%🍁% if you combine text from two or more books, the random
%🍁% text you generate will blend the vocabulary and phrases from
%🍁% the sources in interesting ways.

\item 一旦程序正常运行, 你可以想尝试一下混搭:如果你组合两本或更多书中的文本,
   你生成的随机文本将以有趣的方式混合这些书中的词汇和短语。

\index{mash-up}

\end{enumerate}

%🍁% Credit: This case study is based on an example from Kernighan and
%🍁% Pike, {\em The Practice of Programming}, Addison-Wesley, 1999.

致谢:此案例研究基于 Kernighan 与 Pike 所著的 《{\em The Practice of
Programming}》 一书中的示例。

\end{exercise}

%🍁% You should attempt this exercise before you go on; then you can can
%🍁% download my solution from \url{http://thinkpython2.com/code/markov.py}.
%🍁% You will also need \url{http://thinkpython2.com/code/emma.txt}.

在继续阅读之前, 你应该尝试解决这个习题;
你可以从 \href{http://thinkpython2.com/code/markov.py}{此处}下载我的答案。
你还需要下载\href{http://thinkpython2.com/code/emma.txt}{Emma 的文本} 。

%🍁% \section{Data structures  |  数据结构}
\section{数据结构}
\index{data structure}

%🍁% Using Markov analysis to generate random text is fun, but there is
%🍁% also a point to this exercise: data structure selection.  In your
%🍁% solution to the previous exercises, you had to choose:

使用马尔科夫分析生成随机文本很有趣,
但是上面那道习题的目的是:学习数据结构选择。
在解答上述习题时, 你不得不选择:

%🍁% \begin{itemize}
%🍁% \item How to represent the prefixes.
%🍁% \item How to represent the collection of possible suffixes.
%🍁% \item How to represent the mapping from each prefix to
%🍁% the collection of possible suffixes.
%🍁% \end{itemize}

\begin{itemize}
\item 如何表示前缀。
\item 如何表示可能后缀的集合。
\item 如何表示从前缀到可能后缀集合的映射。
\end{itemize}

%🍁% The last one is easy: a dictionary is the obvious choice
%🍁% for a mapping from keys to corresponding values.

最后一个选择很简单:明显应该选择字典作为键至对应值的映射。

%🍁% For the prefixes, the most obvious options are string,
%🍁% list of strings, or tuple of strings.

对于前缀, 最明显的选择是字符串、字符串列表或者字符串元组。

%🍁% For the suffixes,
%🍁% one option is a list; another is a histogram (dictionary).

对于后缀, 一个选择是列表;另一个是直方图(字典)。

\index{implementation}

%🍁% How should you choose?  The first step is to think about
%🍁% the operations you will need to implement for each data structure.
%🍁% For the prefixes, we need to be able to remove words from
%🍁% the beginning and add to the end.  For example, if the current
%🍁% prefix is ``Half a'', and the next word is ``bee'', you need
%🍁% to be able to form the next prefix, ``a bee''.

你如何选择呢? 第一步是考虑对每个数据结构你需要实现的操作。
对于前缀, 我们需要能从头部删除单词, 并在结尾处加入单词。
例如, 如果当前的前缀是``Half a'', 下一个词是``bee'',
你需要能构成下一个前缀``a bee''。
\index{tuple!as key in dictionary}

%🍁% Your first choice might be a list, since it is easy to add
%🍁% and remove elements, but we also need to be able to use the
%🍁% prefixes as keys in a dictionary, so that rules out lists.
%🍁% With tuples, you can't append or remove, but you can use
%🍁% the addition operator to form a new tuple:

你的第一个选择可能是列表, 因为它能很容易的增加和删除元素,
但是我们也需要让前缀作为字典的键, 这就排除了列表。
使用元组, 你不能追加或删除元素,
但是你能使用加号运算符来形成一个新的元组:

\begin{lstlisting}
def shift(prefix, word):
    return prefix[1:] + (word,)
\end{lstlisting}

%🍁% %
%🍁% {\tt shift} takes a tuple of words, {\tt prefix}, and a string,
%🍁% {\tt word}, and forms a new tuple that has all the words
%🍁% in {\tt prefix} except the first, and {\tt word} added to
%🍁% the end.

\li{shift} 接受一个单词元组 \li{prefix} 和一个字符串 \li{word} ,
并形成一个新的元组, 其具有 \li{prefix} 中除第一个单词外的全部单词,
然后在结尾增加 \li{word} 。

%🍁% For the collection of suffixes, the operations we need to
%🍁% perform include adding a new suffix (or increasing the frequency
%🍁% of an existing one), and choosing a random suffix.

对于后缀的集合, 我们需要执行的运算包括增加一个新的后缀
(或者增加一个已有后缀的频率), 并选择一个随机后缀。

%🍁% Adding a new suffix is equally easy for the list implementation
%🍁% or the histogram.  Choosing a random element from a list
%🍁% is easy; choosing from a histogram is harder to do
%🍁% efficiently (see Exercise~\ref{randhist}).

对于列表或者直方图, 增加一个新的后缀一样容易。
从列表中选择一个随机元素很容易;
在直方图中选择的难度更大(见 练习~\ref{randhist})。

%🍁% So far we have been talking mostly about ease of implementation,
%🍁% but there are other factors to consider in choosing data structures.
%🍁% One is run time.  Sometimes there is a theoretical reason to expect
%🍁% one data structure to be faster than other; for example, I mentioned
%🍁% that the {\tt in} operator is faster for dictionaries than for lists,
%🍁% at least when the number of elements is large.

目前为止, 我们主要讨论实现的难易,
但是选择数据结构时还要考虑其它因素。  一个是运行时间。
有时, 一个数据结构比另一个快有理论依据;
例如, 我提到过 \li{{in} 运算符对于字典比对列表要快,
至少当元素的数目很大的时候。

%🍁% But often you don't know ahead of time which implementation will
%🍁% be faster.  One option is to implement both of them and see which
%🍁% is better.  This approach is called {\bf benchmarking}.  A practical
%🍁% alternative is to choose the data structure that is
%🍁% easiest to implement, and then see if it is fast enough for the
%🍁% intended application.  If so, there is no need to go on.  If not,
%🍁% there are tools, like the {\tt profile} module, that can identify
%🍁% the places in a program that take the most time.

但是通常你事先不知道哪个实现更快。
一个选择是两个都实现, 然后再看哪个更快。
此方法被称作 {\em 基准测试} (benchmarking) 。
另一个更实际的选择是选择最容易实现的数据结构,
然后看它对于拟定的应用是否足够快。  如果是的话, 就不需要继续了。
如果不是, 可以使用一些工具, 如 \li{profile} 模块, 识别程序中哪处最耗时。
\index{benchmarking}
\index{profile module}
\index{module!profile}

%🍁% The other factor to consider is storage space.  For example, using a
%🍁% histogram for the collection of suffixes might take less space because
%🍁% you only have to store each word once, no matter how many times it
%🍁% appears in the text.  In some cases, saving space can also make your
%🍁% program run faster, and in the extreme, your program might not run at
%🍁% all if you run out of memory.  But for many applications, space is a
%🍁% secondary consideration after run time.

另一个要考虑的因素是存储空间。  例如, 使用直方图表示后缀集合可能用更少的空间,
因为无论一个单词在文本中出现多少次, 你只需要存储它一次。
在一些情况下, 节省空间也能让你的程序更快, 极端情况下,
如果内存溢出, 你的程序可能根本不能运行。
但是对于许多应用, 空间是运行时间之后的第二位考虑。

%🍁% One final thought: in this discussion, I have implied that
%🍁% we should use one data structure for both analysis and generation.  But
%🍁% since these are separate phases, it would also be possible to use one
%🍁% structure for analysis and then convert to another structure for
%🍁% generation.  This would be a net win if the time saved during
%🍁% generation exceeded the time spent in conversion.

最后一点:在此讨论中, 我暗示了我们应该使用一种数据结构同时进行分析和生成。
但是既然这些是独立的步骤, 使用一种数据结构进行分析,
然后采用另一种结构进行生成也是可能的。
如果生成节省的时间超过了转化花费的时间, 这也会提高程序的性能。

%🍁% \section{Debugging  |  调试}
\section{调试}
\index{debugging}
%🍁% When you are debugging a program, and especially if you are
%🍁% working on a hard bug, there are five things to try:

在调试一个程序的时候, 特别是调试一个很难的错误时, 应该做到以下五点:

%🍁% \begin{description}
%🍁% \item[Reading:] Examine your code, read it back to yourself, and
%🍁% check that it says what you meant to say.
%🍁% \item[Running:] Experiment by making changes and running different
%🍁% versions.  Often if you display the right thing at the right place
%🍁% in the program, the problem becomes obvious, but sometimes you have to
%🍁% build scaffolding.
%🍁% \item[Ruminating:] Take some time to think!  What kind of error
%🍁% is it: syntax, runtime, or semantic?  What information can you get from
%🍁% the error messages, or from the output of the program?  What kind of
%🍁% error could cause the problem you're seeing?  What did you change
%🍁% last, before the problem appeared?
%🍁% \item[Rubberducking:] If you explain the problem to someone else, you
%🍁%   sometimes find the answer before you finish asking the question.
%🍁%   Often you don't need the other person; you could just talk to a rubber
%🍁%   duck.  And that's the origin of the well-known strategy called {\bf
%🍁%     rubber duck debugging}.  I am not making this up; see
%🍁%   \url{https://en.wikipedia.org/wiki/Rubber_duck_debugging}.
%🍁% \item[Retreating:] At some point, the best thing to do is back
%🍁% off, undoing recent changes, until you get back to a program that
%🍁% works and that you understand.  Then you can start rebuilding.
%🍁% \end{description}

\begin{description}

\item [细读:]检查你的代码, 仔细地阅读, 并且检查是否实现了你的期望。

\item [运行:]通过修改和运行不同的版本来不断试验。    通常, 如果你在程序中正确的地方打印了正确的东西, 问题会变得很明显, 但是有时你不得不搭建一些脚手架。

\item [思考:]    花些时间思考!错误的类型是什么:语法、运行时、语义?
    你从错误信息或者程序的输出中能获得什么信息?
    什么类型的错误能引起你看到的问题?问题出现前, 你最后的修改是什么?

\item [小黄鸭调试法(rubberducking):]如果将你的问题解释给别人听, 有时你会发现在解释完问题之前就能找到答案。
    你通常并不需要真的去问另外一个人;你可以对着一个小黄鸭说。
    这就是著名的{\em 小黄鸭调试法} (rubber duck
    debugging) 的由来。   这可不是我编造的, 看看\href{https://en.wikipedia.org/wiki/Rubber_duck_debugging}{维基的解释}。

\index{维基百科}

\item [回退:]有时候, 最好的做法是回退, 撤销最近的修改,
    直到你回到一个能运行并且你能理解的程序。  然后你可以开始重建。

\end{description}


%🍁% Beginning programmers sometimes get stuck on one of these activities
%🍁% and forget the others.  Each activity comes with its own failure
%🍁% mode.

初级程序员有时陷入这些步骤之一, 忘记了还可以做其他的事情。
事实上, 每种方法都有失败的可能。

\index{typographical error}

%🍁% For example, reading your code might help if the problem is a
%🍁% typographical error, but not if the problem is a conceptual
%🍁% misunderstanding.  If you don't understand what your program does, you
%🍁% can read it 100 times and never see the error, because the error is in
%🍁% your head.

例如, 如果程序是一个排版错误, 读代码可能有帮助,
但是如果问题是概念理解错误, 则未必是这样。
如果你不理解程序要做什么, 可能读100遍程序都不会发现错误, 因为错误在你的头脑中。

\index{experimental debugging}

%🍁% Running experiments can help, especially if you run small, simple
%🍁% tests.  But if you run experiments without thinking or reading your
%🍁% code, you might fall into a pattern I call ``random walk programming'',
%🍁% which is the process of making random changes until the program
%🍁% does the right thing.  Needless to say, random walk programming
%🍁% can take a long time.

试验可能会有帮助, 特别是如果你运行简单短小的测试。
但是, 如果你不思考或者阅读你的代码, 就直接进行实验,
你可能陷入一种我称为“随机游走编程”的模式。
这指的是随机修改, 直到程序通过测试。
不用说, 随机游走编程会花费很长的时间。

\index{random walk programming}
\index{development plan!random walk programming}

%🍁% You have to take time to think.  Debugging is like an
%🍁% experimental science.  You should have at least one hypothesis about
%🍁% what the problem is.  If there are two or more possibilities, try to
%🍁% think of a test that would eliminate one of them.

你必须花时间思考。  调试就像是一门实验科学。
你应该至少有一个关于问题是什么的假设。
如果有两个或者更多的可能, 试着考虑利用测试消除其中一个可能。

%🍁% But even the best debugging techniques will fail if there are too many
%🍁% errors, or if the code you are trying to fix is too big and
%🍁% complicated.  Sometimes the best option is to retreat, simplifying the
%🍁% program until you get to something that works and that you
%🍁% understand.

但是, 如果有太多的错误, 或者你正试图修复的代码太大、太复杂,
即使最好的调试技巧也会失败。
有时, 最好的选择是回退, 简化程序, 直到你获得一个正常运行并且能理解的程序。

%🍁% Beginning programmers are often reluctant to retreat because
%🍁% they can't stand to delete a line of code (even if it's wrong).
%🍁% If it makes you feel better, copy your program into another file
%🍁% before you start stripping it down.  Then you can copy the pieces
%🍁% back one at a time.

初级程序员经常不愿意回退, 因为他们舍不得删除一行代码(即使它是错误的)。
如果能让你好受些, 在你开始精简之前, 可以将你的代码拷贝到另一个文件中。
然后你再把修改后的代码一块一块地拷贝回去。

%🍁% Finding a hard bug requires reading, running, ruminating, and
%🍁% sometimes retreating.  If you get stuck on one of these activities,
%🍁% try the others.

发现一个错误, 需要阅读、运行、沉思、和时而的回退。
如果其中某个步骤没有进展, 试一下其它的。

%🍁% \section{Glossary  |  术语表}
\section{术语表}
\begin{description}
%🍁% \item[deterministic:] Pertaining to a program that does the same
%🍁% thing each time it runs, given the same inputs.

\item[确定性的 (deterministic):] 指的是给定相同的输入, 一个程序每次运行的结果是一样的。
\index{deterministic}

%🍁% \item[pseudorandom:] Pertaining to a sequence of numbers that appears
%🍁% to be random, but is generated by a deterministic program.

\item[伪随机 (pseudorandom):] 指的是一串数字看上去是随机的, 但是实际是由一个确定性程序生成的。
\index{pseudorandom}

%🍁% \item[default value:] The value given to an optional parameter if no
%🍁% argument is provided.

\item[默认值:] 没有提供实参时, 赋给可选形参的值。
\index{default value}

%🍁% \item[override:] To replace a default value with an argument.

\item[覆盖:] 用实参替代默认值。
\index{override}

%🍁% \item[benchmarking:] The process of choosing between data structures
%🍁% by implementing alternatives and testing them on a sample of the
%🍁% possible inputs.

\item[基准测试 (benchmarking):] 通过可能的输入样本对使用不同数据结构的实现进行测试, 从而选择数据结构的过程。
\index{benchmarking}

%🍁% \item[rubber duck debugging:] Debugging by explaining your problem
%🍁% to an inanimate object such as a rubber duck.  Articulating the
%🍁% problem can help you solve it, even if the rubber duck doesn't know
%🍁% Python.

\item[小黄鸭调试法 (rubberducking):] 通过向小黄鸭这样的非生物体解释你的问题来进行调试。
    清晰地陈述问题可以帮助你解决问题, 即使小黄鸭并不懂 Python。

\index{rubber duck debugging}
\index{debugging!rubber duck}
\end{description}

%🍁% \section{Exercises  |  练习}
\section{练习}
\begin{exercise}
\index{word frequency}
\index{frequency!word}
\index{Zipf's law}

%🍁% The ``rank'' of a word is its position in a list of words
%🍁% sorted by frequency: the most common word has rank 1, the
%🍁% second most common has rank 2, etc.

单词的``秩''是指它在按照单词频率排序的列表中的位置:
出现频率最高的单词, 它的秩是{\em 1}, 频率第二高的单词, 它的秩是{\em 2}, 以此类推。

%🍁% Zipf's law describes a relationship between the ranks and frequencies
%🍁% of words in natural languages
%🍁% (\url{http://en.wikipedia.org/wiki/Zipf's_law}).  Specifically, it
%🍁% predicts that the frequency, $f$, of the word with rank $r$ is:

\href{http://en.wikipedia.org/wiki/Zipf's_law}{Zipf定律} 描述了自然语言中秩和单词出现频率的关系。  特别是, 它预测对于秩为 $r$ 的单词,
其出现的频率 $f$ 是:

\[ f = c r^{-s} \]

%🍁% %
%🍁% where $s$ and $c$ are parameters that depend on the language and the
%🍁% text.  If you take the logarithm of both sides of this equation, you
%🍁% get:

其中, $s$ 和 $c$ 是依赖于语言和文本的参数。  如果在上述等式两边取对数的话, 你可以得到:
\index{logarithm}

\[ \log f = \log c - s \log r \]

%🍁% %
%🍁% So if you plot log $f$ versus log $r$, you should get
%🍁% a straight line with slope $-s$ and intercept log $c$.

因此, 如果绘出 {\em log} $f$ 和 {\em log} $r$ 的图像, 你可以得到一条以 $-s$ 为斜率、以 $c$ 为截距的直线。

%🍁% Write a program that reads a text from a file, counts
%🍁% word frequencies, and prints one line
%🍁% for each word, in descending order of frequency, with
%🍁% log $f$ and log $r$.  Use the graphing program of your
%🍁% choice to plot the results and check whether they form
%🍁% a straight line.  Can you estimate the value of $s$?

编写一个程序, 从文件中读取文本, 计算单词频率, 倒序输出每个单词,
一个单词一行, 同时在这行输出对应的 {\em log} $f$ 和 {\em log} $r$。
使用你喜欢的绘图程序, 画出结果并检查是不是形成一条直线。
你可以估算出 $s$ 的值吗?

%🍁% Solution: \url{http://thinkpython2.com/code/zipf.py}.
%🍁% To run my solution, you need the plotting module {\tt matplotlib}.
%🍁% If you installed Anaconda, you already have {\tt matplotlib};
%🍁% otherwise you might have to install it.
\index{matplotlib}

\href{http://thinkpython2.com/code/zipfpy}{参考答案}

你需要安装绘图模块 \href{http://www.matplotlib.org}{{\em \li{matplotlib}}} 才能运行我的答案。
当然如果你安装了  \href{http://www.anaconda.org}{{\em \li{Anaconda}}}, 你就已经有了 {\em \li{matplotlib}};否则你需要安装它。
\index{matplotlib}

\end{exercise}