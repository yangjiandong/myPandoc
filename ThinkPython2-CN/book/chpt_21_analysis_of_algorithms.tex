

%🍁% \chapter{Analysis of Algorithms  |  算法分析}
\chapter{算法分析}
\label{algorithms}

%🍁% \begin{quote}
%🍁% This appendix is an edited excerpt from {\it Think Complexity}, by
%🍁% Allen B. Downey, also published by O'Reilly Media (2012).  When you
%🍁% are done with this book, you might want to move on to that one.
%🍁% \end{quote}

\begin{quote}
本附录摘自 Allen B. Downey的 {\em Think Complexity}, 也由 O’Reilly Media (2011)出版。 当你读完本书后,也许你可以接着读读那本书。
\end{quote}

%🍁% {\bf Analysis of algorithms} is a branch of computer science that
%🍁% studies the performance of algorithms, especially their run time and
%🍁% space requirements.  See
%🍁% \url{http://en.wikipedia.org/wiki/Analysis_of_algorithms}.

{\em 算法分析} ({\bf Analysis of algorithms}) 是计算机科学的一个分支,
 着重研究算法的性能, 特别是他们的运行时间和资源开销。
见:\href{http://en.wikipedia.org/wiki/Analysis_of_algorithms}{算法分析\textsuperscript{(维基百科)}} 。

\index{algorithm} \index{analysis of algorithms}

\index{维基百科}


%🍁% The practical goal of algorithm analysis is to predict the performance
%🍁% of different algorithms in order to guide design decisions.

算法分析的实际目的是预测不同算法的性能,用于指导设计决策。

%🍁% During the 2008 United States Presidential Campaign, candidate
%🍁% Barack Obama was asked to perform an impromptu analysis when
%🍁% he visited Google.  Chief executive Eric Schmidt jokingly asked him
%🍁% for ``the most efficient way to sort a million 32-bit integers.''
%🍁% Obama had apparently been tipped off, because he quickly
%🍁% replied, ``I think the bubble sort would be the wrong way to go.''
%🍁% See \url{http://www.youtube.com/watch?v=k4RRi_ntQc8}.

2008年美国总统大选期间,当候选人 奥巴马(Barack Obama) 访问 Google 时,
他被要求进行即席的分析。  首席执行官Eric Schmidt开玩笑的问他
``对一百万个32位整数排序的最有效的方法''。
显然有人暗中通知了奥巴马,因为他很快回答, ``我认为不应该采用冒泡排序法''。
看看\href{http://www.youtube.com/watch?v=k4RRi_ntQc8}{这个视频}。

\index{Obama, Barack}  \index{Schmidt, Eric}  \index{bubble sort}

%🍁% This is true: bubble sort is conceptually simple but slow for
%🍁% large datasets.  The answer Schmidt was probably looking for is
%🍁% ``radix sort'' (\url{http://en.wikipedia.org/wiki/Radix_sort})\footnote{
%🍁% But if you get a question like this in an interview, I think
%🍁% a better answer is, ``The fastest way to sort a million integers
%🍁% is to use whatever sort function is provided by the language
%🍁% I'm using.  Its performance is good enough for the vast majority
%🍁% of applications, but if it turned out that my application was too
%🍁% slow, I would use a profiler to see where the time was being
%🍁% spent.  If it looked like a faster sort algorithm would have
%🍁% a significant effect on performance, then I would look
%🍁% around for a good implementation of radix sort.''}.
\index{radix sort}

是真的:冒泡排序概念上很简单,但是对于大数据集速度非常慢。
Schmidt 所提问的答案可能是 ``基数排序 (\href{http://en.wikipedia.org/wiki/Radix_sort}{radix sort})''\footnote{但是,如果你采访中被问到这个问题,更好的答案可能是,``对上百万个整数的最快的排序方法就是用你所使用的语言的内建排序函数。 它的性能对于大多数应用而言已优化的足够好。但如果是我自己写的排序程序运行太慢, 我会用性能分析器找出大量的运算时间被用在了哪儿。如果一个更快的算法会对性能产生显著的提升,我会先试试基数排序。''}。

%🍁% The goal of algorithm analysis is to make meaningful
%🍁% comparisons between algorithms, but there are some problems:

算法分析的目的是在不同算法间进行有意义的比较, 但是有一些问题:
\index{comparing algorithms}

\begin{itemize}

%🍁% \item The relative performance of the algorithms might
%🍁% depend on characteristics of the hardware, so one algorithm
%🍁% might be faster on Machine A, another on Machine B.
%🍁% The general solution to this problem is to specify a
%🍁% {\bf machine model} and analyze the number of steps, or
%🍁% operations, an algorithm requires under a given model.
%🍁% \index{machine model}

\item 算法相对的性能依赖于硬件的特性,因此一个算法可能在机器A上比较快, 另一个算法则在机器B上比较快。
对此问题一般的解决办法是指定一个 {\em 机器模型} (machine model) 并且分析一个算法在一个给定模型下所需的步骤或运算的数目。
\index{machine model} \index{机器模型}

%🍁% \item Relative performance might depend on the details of
%🍁% the dataset.  For example, some sorting
%🍁% algorithms run faster if the data are already partially sorted;
%🍁% other algorithms run slower in this case.
%🍁% A common way to avoid this problem is to analyze the
%🍁% {\bf worst case} scenario.  It is sometimes useful to
%🍁% analyze average case performance, but that's usually harder,
%🍁% and it might not be obvious what set of cases to average over.
%🍁% \index{worst case}  \index{average case}

\item 相对性能可能依赖于数据集的细节。
例如, 如果数据已经部分排好序, 一些排序算法可能更快; 此时其它算法运行的比较慢。
避免该问题的一般方法是分析 {\em 最坏情况}。
有时分析平均情况性能, 但那通常更难, 而且可能不容易弄清该对哪些数据集合进行平均。
\index{worst case}  \index{average case}

%🍁% \item Relative performance also depends on the size of the
%🍁% problem.  A sorting algorithm that is fast for small lists
%🍁% might be slow for long lists.
%🍁% The usual solution to this problem is to express run time
%🍁% (or number of operations) as a function of problem size,
%🍁% and group functions into categories depending on how quickly
%🍁% they grow as problem size increases.

\item 相对性能也依赖于问题的规模。
一个对于小列表很快的排序算法可能对于长列表很慢。
对此问题通常的解决方法是将运行时间(或者运算的数目)表示成问题规模的函数, 并且根据各自随着问题规模的增长而增加的速度,将函数分成不同的类别。
\end{itemize}

%🍁% The good thing about this kind of comparison is that it lends
%🍁% itself to simple classification of algorithms.  For example,
%🍁% if I know that the run time of Algorithm A tends to be
%🍁% proportional to the size of the input, $n$, and Algorithm B
%🍁% tends to be proportional to $n^2$, then I
%🍁% expect A to be faster than B, at least for large values of $n$.

关于此类比较的好处是有助于对算法进行简单的分类。
例如,如果我知道算法 A 的运行时间与输入的规模 $n$ 成正比,
算法 B 与 $n^2$ 成正比,那么我 可以认为A 比 B 快, 至少对于很大的 $n$ 值来说是这样。

%🍁% This kind of analysis comes with some caveats, but we'll get
%🍁% to that later.

这类分析也有一些问题,后面会作提及。


%🍁% \section{Order of growth  |  增长量级}
\section{增长量级}

%🍁% Suppose you have analyzed two algorithms and expressed
%🍁% their run times in terms of the size of the input:
%🍁% Algorithm A takes $100n+1$ steps to solve a problem with
%🍁% size $n$; Algorithm B takes $n^2 + n + 1$ steps.

假设你已经分析了两个算法,并能用输入计算量的规模表示它们的运行时间:
若算法 A 用 $100n+1$ 步解决一个规模为 $n$ 的问题;
而算法 B 用 $n^2 + n + 1$ 步。
\index{order of growth}  \index{增长的阶数}

%🍁% The following table shows the run time of these algorithms
%🍁% for different problem sizes:

下表列出了这些算法对于不同问题规模的运行时间:

\begin{center}
\begin{tabular}{|r|r|r|}
\hline
Input     &   Run time of     & Run time of \\
size      &   Algorithm A     & Algorithm B \\
\hline
10        &   1 001           & 111         \\
100       &   10 001          & 10 101         \\
1 000     &   100 001         & 1 001 001         \\
10 000    &   1 000 001       & $> 10^{10}$         \\
\hline
\end{tabular}
\end{center}

%🍁% At $n=10$, Algorithm A looks pretty bad; it takes almost 10 times
%🍁% longer than Algorithm B.  But for $n=100$ they are about the same, and
%🍁% for larger values A is much better.

当 $n=10$ 时,算法 A 看上去很糟糕,它用了 10 倍于算法 B 所需的时间。
但当 $n=100$ 时 ,它们性能几乎相同, 而 $n$ 取更大值时,算法 A 要好得多。

%🍁% The fundamental reason is that for large values of $n$, any function
%🍁% that contains an $n^2$ term will grow faster than a function whose
%🍁% leading term is $n$.  The {\bf leading term} is the term with the
%🍁% highest exponent.

根本原因是对于较大的 $n$ 值,任何包含 $n^2$ 项的函数都比首项为 $n$ 的函数增长要快。
{\em 首项} (leading term) 是指具有最高指数的项。
\index{leading term}  \index{exponent}

%🍁% For Algorithm A, the leading term has a large coefficient, 100, which
%🍁% is why B does better than A for small $n$.  But regardless of the
%🍁% coefficients, there will always be some value of $n$ where
%🍁% $a n^2 > b n$, for any values of $a$ and $b$.

对于算法A,首项有一个较大的系数 100,这就是为什么对于小 $n$ ,B 比 A 好。
但是不考虑该系数,总有一些 $n$ 值使得 $a n^2 > b n$。
\index{leading coefficient}

%🍁% The same argument applies to the non-leading terms.  Even if the run
%🍁% time of Algorithm A were $n+1000000$, it would still be better than
%🍁% Algorithm B for sufficiently large $n$.

同样的理由适用于非首项。
即使算法 A 的运行时间为 $n+1000000$ ,对于足够大的 $n$ ,它仍然比算法 B 好。

%🍁% In general, we expect an algorithm with a smaller leading term to be a
%🍁% better algorithm for large problems, but for smaller problems, there
%🍁% may be a {\bf crossover point} where another algorithm is better.  The
%🍁% location of the crossover point depends on the details of the
%🍁% algorithms, the inputs, and the hardware, so it is usually ignored for
%🍁% purposes of algorithmic analysis.  But that doesn't mean you can forget
%🍁% about it.

一般来讲,我们认为具备较小首项的算法对于规模大的问题是一个好算法,但是对于规模小的问题,可能存在一个 {\em 交叉点} (crossover point) , 在此规模以下,另一个算法更好。  交叉点的位置取决于算法的细节、 输入以及硬件,因此对于算法分析目的,它通常被忽略。
但是这不意味着你可以忘记它的存在。
\index{crossover point}

%🍁% If two algorithms have the same leading order term, it is hard to say
%🍁% which is better; again, the answer depends on the details.  So for
%🍁% algorithmic analysis, functions with the same leading term
%🍁% are considered equivalent, even if they have different coefficients.

如果两个算法有相同的首项,很难说哪个更好。 答案还是取决于细节。
所以,对于算法分析来说,具有相同首项的函数被认为是相当的,即使它们具有不同的系数。

%🍁% An {\bf order of growth} is a set of functions whose growth
%🍁% behavior is considered equivalent.  For example, $2n$, $100n$ and $n+1$
%🍁% belong to the same order of growth, which is written $O(n)$ in
%🍁% {\bf Big-Oh notation} and often called {\bf linear} because every function
%🍁% in the set grows linearly with $n$.

{\em 增长量级}\footnote{译注:又译{\em 增长级},{\em 增长阶数}; 即:算法性能} (order of growth) 是一个函数集合, 集合中函数的增长行为被认为是相当的。
例如 $2n$ 、 $100n$ 和 $n+1$ 属于相同的增长量级,
用 \href{https://zh.wikipedia.org/wiki/%E5%A4%A7O%E7%AC%A6%E5%8F%B7}{{\em 大{\em O}符号}} (Big-Oh notation)表示, 写成 $O(n)$ ,
而且通常被称作 {\em 线性的} (linear) ,
因为集合中的每个函数根据 $n$ 线性增长。
\index{big-oh notation}  \index{linear growth}

%🍁% All functions with the leading term $n^2$ belong to $O(n^2)$; they are
%🍁% called {\bf quadratic}.

首项为 $n^2$ 的函数属于 $O(n^2)$;  它们是 {\em 二次方级} (quadratic) ,

\index{quadratic growth}

%🍁% The following table shows some of the orders of growth that
%🍁% appear most commonly in algorithmic analysis,
%🍁% in increasing order of badness.

下表按照计算性能开销效率降序顺序排列显示了算法分析中最通常的一些增长量级。
\index{badness}

%🍁%  \begin{tabular}{|r|r|r|}
%🍁%  \hline
%🍁%  Order of     &   Name      \\
%🍁%  growth       &               \\
%🍁%  \hline
%🍁%  $O(1)$             & constant \\
%🍁%  $O(\log_b n)$      & logarithmic (for any $b$) \\
%🍁%  $O(n)$             & linear \\
%🍁%  $O(n \log_b n)$    & linearithmic \\
%🍁%  $O(n^2)$           & quadratic     \\
%🍁%  $O(n^3)$           & cubic     \\
%🍁%  $O(c^n)$           & exponential (for any $c$)    \\
%🍁%  \hline
%🍁%  \end{tabular}

\begin{center}
\begin{tabular}{|r|r|r|}
\hline
增长量级     &   名称      \\
\hline
$O(1)$             & 常数级 (constant) \\
$O(\log_b n)$      & 对数级 (logarithmic) (对于任意 $b$) \\
$O(n)$             & 线性级 (linear)\\
$O(n \log_b n)$    & 线性对数级 (linearithmic)\\
$O(n^2)$           & 二次方级 (quadratic)    \\
$O(n^3)$           & 三次方级 (cubic)    \\
$O(c^n)$           & 指数级 (exponential) (对于任意 $c$)    \\
\hline
\end{tabular}
\end{center}

%🍁% For the logarithmic terms, the base of the logarithm doesn't matter;
%🍁% changing bases is the equivalent of multiplying by a constant, which
%🍁% doesn't change the order of growth.  Similarly, all exponential
%🍁% functions belong to the same order of growth regardless of the base of
%🍁% the exponent. Exponential functions grow very quickly,
%🍁% so exponential algorithms are only useful for small problems.

对于对数级数, 对数的基数并不影响增长量级。
改变阶数等价于乘以一个常数, 其不改变增长量级。
相应的, 所有的指数级数都属于相同的增长量级, 而无需考虑指数的基数大小
指数函数增长量级增长的非常快, 因此指数级算法只用于小规模问题。
\index{logarithmic growth}  \index{exponential growth}


\begin{exercise}

%🍁% Read the Wikipedia page on Big-Oh notation at
%🍁% \url{http://en.wikipedia.org/wiki/Big_O_notation} and
%🍁% answer the following questions:

阅读维基百科关于 \href{http://en.wikipedia.org/wiki/Big_O_notation}{大O符号} 的介绍,回答以下问题:

\index{维基百科}

\begin{enumerate}
%🍁% \item What is the order of growth of $n^3 + n^2$?
%🍁% What about $1000000 n^3 + n^2$?
%🍁% What about $n^3 + 1000000 n^2$?

\item $n^3 + n^2$ 的增长量级是多少?  $1000000 n^3 + n^2$ 和 $n^3 + 1000000 n^2$ 的增长量级又是多少?

%🍁% \item What is the order of growth of $(n^2 + n) \cdot (n + 1)$?  Before
%🍁%   you start multiplying, remember that you only need the leading term.

\item $(n^2 + n) \cdot (n + 1)$ 的增长量级是多少?  在开始计算之前,记住你只需要考虑首项即可。

%🍁% \item If $f$ is in $O(g)$, for some unspecified function $g$, what can
%🍁%   we say about $af+b$?

\item 如果 $f$ 的增长量级为 $O(g)$ ,那么对于未指定的函数 $g$ ,我们可以如何描述 $af+b$ ?

%🍁% \item If $f_1$ and $f_2$ are in $O(g)$, what can we say about $f_1 + f_2$?

\item 如果 $f_1$ 和 $f_2$ 的增长量级为 $O(g)$,那么 $f_1 + f_2$ 的增长量级又是多少?

%🍁% \item If  $f_1$ is in $O(g)$
%🍁% and $f_2$ is in $O(h)$,
%🍁% what can we say about  $f_1 + f_2$?

\item 如果 $f_1$ 的增长量级为 $O(g)$ ,$f_2$ 的增长量级为 $O(h)$,那么 $f_1 + f_2$ 的增长量级是多少?

%🍁% \item If  $f_1$ is in $O(g)$ and $f_2$ is $O(h)$,
%🍁% what can we say about  $f_1 \cdot f_2$?
\end{enumerate}

\end{exercise}

%🍁% Programmers who care about performance often find this kind of
%🍁% analysis hard to swallow.  They have a point: sometimes the
%🍁% coefficients and the non-leading terms make a real difference.
%🍁% Sometimes the details of the hardware, the programming language, and
%🍁% the characteristics of the input make a big difference.  And for small
%🍁% problems asymptotic behavior is irrelevant.

关注性能的程序员经常发现这种分析很难忍受。  他们有一个观点:有时系数和非首项会造成巨大的影响。  有时,硬件的细节、编程语言以及输入的特性会造成很大的影响。  对于小问题,渐近的行为没有什么影响。

%🍁% But if you keep those caveats in mind, algorithmic analysis is a
%🍁% useful tool.  At least for large problems, the ``better'' algorithms
%🍁% is usually better, and sometimes it is {\em much} better.  The
%🍁% difference between two algorithms with the same order of growth is
%🍁% usually a constant factor, but the difference between a good algorithm
%🍁% and a bad algorithm is unbounded!

但是,如果你记得那些警告,算法分析就是一个有用的工具。
至少对于大问题 ``更好的'' 算法通常更好,并且有时它要好的多。
相同增长量级的两个算法之间的不同通常是一个常数因子,
但是一个好算法和一个坏算法之间的不同是无限的!


%🍁% \section{Analysis of basic Python operations  |  Python基本运算操作分析}
\section{Python基本运算操作分析}

%🍁% In Python, most arithmetic operations are constant time;
%🍁% multiplication usually takes longer than addition and subtraction, and
%🍁% division takes even longer, but these run times don't depend on the
%🍁% magnitude of the operands.  Very large integers are an exception; in
%🍁% that case the run time increases with the number of digits.

在 Python 中,大部分算术运算的开销是常数级的。
乘法会比加减法用更长的时间,除法更长,但是这些运算时间不依赖被运算数的数量级。
非常大的整数却是个例外,在这种情况下,运行时间随着位数的增加而增加。
\index{analysis of primitives}

%🍁% Indexing operations---reading or writing elements in a sequence
%🍁% or dictionary---are also constant time, regardless of the size
%🍁% of the data structure.

索引操作 --- 在序列或字典中读写元素 --- 在不考虑数据结构的大小的情况下也是常数级的。
\index{indexing}

%🍁% A {\tt for} loop that traverses a sequence or dictionary is
%🍁% usually linear, as long as all of the operations in the body
%🍁% of the loop are constant time.  For example, adding up the
%🍁% elements of a list is linear:

一个遍历序列或字典的 \li{for} 循环通常是线性的, 只要循环体内的运算是常数时间。
例如,累加一个列表的元素是线性的:

\begin{lstlisting}
    total = 0
    for x in t:
        total += x
\end{lstlisting}

%🍁% The built-in function {\tt sum} is also linear because it does
%🍁% the same thing, but it tends to be faster because it is a more
%🍁% efficient implementation; in the language of algorithmic analysis,
%🍁% it has a smaller leading coefficient.

内建求和函数 \li{sum} 也是线性的,因为它做相同的事情,
但是它倾向于更快因为它是一个更有效的实现。
从算法分析角度讲,它具有更小的首项系数。

%🍁% As a rule of thumb, if the body of a loop is in $O(n^a)$ then
%🍁% the whole loop is in $O(n^{a+1})$.  The exception is if you can
%🍁% show that the loop exits after a constant number of iterations.
%🍁% If a loop runs $k$ times regardless of $n$, then
%🍁% the loop is in $O(n^a)$, even for large $k$.

根据经验, 如果循环体内的增长量级是 $O(n^a)$, 则整个循环的增长量级是 $O(n^{a+1})$。
如果这个循环在执行一定数目循环后退出则是例外。 无论 $n$ 取值多少, 如果循环仅执行 $k$ 次, 整个循环的增长量级是 $O(n^a)$, 即便 $k$ 值比较大。

%🍁% Multiplying by $k$ doesn't change the order of growth, but neither
%🍁% does dividing.  So if the body of a loop is in $O(n^a)$ and it runs
%🍁% $n/k$ times, the loop is in $O(n^{a+1})$, even for large $k$.

乘上 $k$ 并不会改变增长量级,除法也是。 因此,如果循环体的增长量级是 $O(n^a)$, 循环执行 $n/k$ 次, 那么整个循环的增长量级就是 $O(n^{a+1})$ , 即使 $k$ 值很大。

%🍁% Most string and tuple operations are linear, except indexing and {\tt
%🍁%   len}, which are constant time.  The built-in functions {\tt min} and
%🍁% {\tt max} are linear.  The run-time of a slice operation is
%🍁% proportional to the length of the output, but independent of the size
%🍁% of the input.

大部分字符串和元组运算是线性的,除了索引和 \li{len},它们是常数时间。
内建函数 \li{min} 和 \li{max} 是线性的。
切片运算与输出的长度成正比,但是和输入的大小无关。
\index{string methods}  \index{tuple methods}
\index{字符串方法}  \index{元组方法}

%🍁% String concatenation is linear; the run time depends on the sum
%🍁% of the lengths of the operands.

字符串拼接是线性的; 它的运算时间取决于 运算对象 的总长度。
\index{string concatenation}

%🍁% All string methods are linear, but if the lengths of
%🍁% the strings are bounded by a constant---for example, operations on single
%🍁% characters---they are considered constant time.
%🍁% The string method {\tt join} is linear; the run time depends on
%🍁% the total length of the strings.

所有字符串方法是线性的,但是如果字符串的长度受限于一个常数 --- 例如,
在单个字符上的运算 --- 它们被认为是常数时间。
字符串方法 \li{join} 也是线性的; 它的运算时间取决于字符串的总长度。

\index{join@{\tt join}}

%🍁% Most list methods are linear, but there are some exceptions:

大部分列表方法是线性的,但是有一些例外:
\index{list methods}

\begin{itemize}

%🍁% \item Adding an element to the end of a list is constant time on
%🍁% average; when it runs out of room it occasionally gets copied
%🍁% to a bigger location, but the total time for $n$ operations
%🍁% is $O(n)$, so the average time for each
%🍁% operation is $O(1)$.

\item 平均来讲,在列表结尾增加一个元素是常数时间。 当它超出了所占用空间时,它偶尔被拷贝到一个更大的地方, 但是对于 $n$ 个运算的整体时间仍为 $O(n)$ , 所以我们说一个运算的“分摊”时间是 $O(1)$ 。

%🍁% \item Removing an element from the end of a list is constant time.

\item 从一个列表结尾删除一个元素是常数时间。

%🍁% \item Sorting is $O(n \log n)$.

\item 排序是 $O(n \log n)$ 。
\index{sorting}  \index{排序}

\end{itemize}

%🍁% Most dictionary operations and methods are constant time, but
%🍁% there are some exceptions:

大部分字典运算和方法是常数时间,但有些例外:
\index{dictionary methods}

\begin{itemize}

%🍁% \item The run time of {\tt update} is
%🍁%   proportional to the size of the dictionary passed as a parameter,
%🍁%   not the dictionary being updated.

\item \li{update} 的运行时间与作为形参被传递的字典(不是被更新的字典)的大小成正比。

%🍁% \item {\tt keys}, {\tt values} and {\tt items} are constant time because
%🍁%   they return iterators.  But if you loop through the iterators, the loop will be linear.

\item \li{keys}、 \li{values} 和 \li{items} 是常数时间,因为它们返回迭代器。
   但是如果你对迭代器进行循环,循环将是线性的。
\index{iterator}

\end{itemize}

%🍁% The performance of dictionaries is one of the minor miracles of
%🍁% computer science.  We will see how they work in
%🍁% Section~\ref{hashtable}.

字典的性能是计算机科学的一个小奇迹之一。
在 \hyperref[hashtable]{哈希表} 一节中,我们将看到它们是如何工作的。

\begin{exercise}

%🍁% Read the Wikipedia page on sorting algorithms at
%🍁% \url{http://en.wikipedia.org/wiki/Sorting_algorithm} and answer
%🍁% the following questions:
%🍁% \index{sorting}

阅读\href{http://en.wikipedia.org/wiki/Sorting_algorithm}{排序算法}在维基百科的介绍,回答下面的问题:
\index{sorting}
\index{维基百科}


\begin{enumerate}

%🍁% \item What is a ``comparison sort?'' What is the best worst-case order
%🍁%   of growth for a comparison sort?  What is the best worst-case order
%🍁%   of growth for any sort algorithm?

\item 什么是 ``\href{https://zh.wikipedia.org/wiki/%E6%AF%94%E8%BE%83%E6%8E%92%E5%BA%8F}{比较排序} (\href{https://en.wikipedia.org/wiki/Comparison_sort}{comparison sort})''?  比较排序最优最差情况的增长量级是多少?  别的排序算法的 最优最差情况的增长量级 又是多少?
\index{comparison sort}

%🍁% \item What is the order of growth of bubble sort, and why does Barack
%🍁%   Obama think it is ``the wrong way to go?''

\item 冒泡排序法的增长量级是多少? 为什么奥巴马认为是``不应采用的方法''

%🍁% \item What is the order of growth of radix sort?  What preconditions
%🍁%   do we need to use it?

\item 基数排序(radix sort)的增长量级是多少? 我们使用它所需要注意的前提条件有哪些?

%🍁% \item What is a stable sort and why might it matter in practice?

\item 排序算法的稳定性是指什么?为什么它在实际操作中很重要?
\index{stable sort}

%🍁% \item What is the worst sorting algorithm (that has a name)?

\item 最差的排序算法是哪一个(有名称的)?

%🍁% \item What sort algorithm does the C library use?  What sort algorithm
%🍁%   does Python use?  Are these algorithms stable?  You might have to
%🍁%   Google around to find these answers.

\item C 语言使用哪种排序算法? Python使用哪种排序算法? 这些算法稳定吗? 你可能需要谷歌一下,才能找到这些答案。

%🍁% \item Many of the non-comparison sorts are linear, so why does does
%🍁%   Python use an $O(n \log n)$ comparison sort?

\item 大多数 非比较 算法是线性的, 因此为什么 Python 使用一个 $O(n \log n)$ 的 comparison sort ?

\end{enumerate}

\end{exercise}


%🍁% \section{Analysis of search algorithms  |  搜索算法分析}
\section{搜索算法分析}

%🍁% A {\bf search} is an algorithm that takes a collection and a target
%🍁% item and determines whether the target is in the collection, often
%🍁% returning the index of the target.

{ \em 搜索} (search) 算法,其接受一个集合以及一个目标项,
并决定该目标项是否在集合中,通常返回目标的索引值。
\index{search}  \index{搜索}

%🍁% The simplest search algorithm is a ``linear search'', which traverses
%🍁% the items of the collection in order, stopping if it finds the target.
%🍁% In the worst case it has to traverse the entire collection, so the run
%🍁% time is linear.

最简单的搜素算法是``线性搜索'', 其按顺序遍历集合中的项, 如果找到目标则停止。 最坏的情况下, 它不得不遍历全部集合, 所以运行时间是线性的。
\index{linear search}

%🍁% The {\tt in} operator for sequences uses a linear search; so do string
%🍁% methods like {\tt find} and {\tt count}.

序列的 in 运算符使用线性搜索。 字符串方法,如 \li{find} 和 \li{count} 也是这样。
\index{in@{\tt in} operator}

%🍁% If the elements of the sequence are in order, you can use a {\bf
%🍁%   bisection search}, which is $O(\log n)$.  Bisection search is
%🍁% similar to the algorithm you might use to look a word up in a
%🍁% dictionary (a paper dictionary, not the data structure).  Instead of
%🍁% starting at the beginning and checking each item in order, you start
%🍁% with the item in the middle and check whether the word you are looking
%🍁% for comes before or after.  If it comes before, then you search the
%🍁% first half of the sequence.  Otherwise you search the second half.
%🍁% Either way, you cut the number of remaining items in half.

如果元素在序列中是排序好的,你可以用 {\em 二分搜素} (bisection search) ,
它的增长量级是 $O(\log n)$ 。  二分搜索和你在字典中查找一个单词的算法类似
(这里是指真正的字典,不是数据结构)。 你不会从头开始并按顺序检查每个项,
而是从中间的项开始并检查你要查找的单词在前面还是后面。
如果它出现在前面,那么你搜索序列的前半部分。
否则你搜索后一半。 如论如何,你将剩余的项数分为一半。
\index{bisection search}  \index{二分搜素}

%🍁% If the sequence has 1,000,000 items, it will take about 20 steps to
%🍁% find the word or conclude that it's not there.  So that's about 50,000
%🍁% times faster than a linear search.

如果序列有 1,000,000 项,它将花 20 步找到该单词或说找不到。
因此它比线性搜索快大概 50,000 倍。

%🍁% Bisection search can be much faster than linear search, but
%🍁% it requires the sequence to be in order, which might require
%🍁% extra work.

二分搜索比线性搜索快很多,但是它要求已排序的序列,因此使用时会有额外的工作要做。

%🍁% There is another data structure, called a {\bf hashtable} that
%🍁% is even faster---it can do a search in constant time---and it
%🍁% doesn't require the items to be sorted.  Python dictionaries
%🍁% are implemented using hashtables, which is why most dictionary
%🍁% operations, including the {\tt in} operator, are constant time.

另一个检索速度更快的数据结构被称为 {\em 哈希表} (hashtable) --- 它可以在 常数时间 内检索出结果 --- 并且不依赖于序列是否已排序。 Python 的内建字典就通过哈希表技术而实现,因此大多数的字典操作,包括 \li{in} ,只花费常数时间就可完成。


%🍁% \section{Hashtables  |  哈希表}
\section{哈希表}
\label{hashtable}

%🍁% To explain how hashtables work and why their performance is so
%🍁% good, I start with a simple implementation of a map and
%🍁% gradually improve it until it's a hashtable.

为了解释哈希表是如何工作,以及为什么它的性能非常优秀,
我们从实现一个简单的映射 (map) 开始并逐步改进它,直到成为一个哈希表。
\index{hashtable}  \index{哈希表}

%🍁% I use Python to demonstrate these implementations, but in real
%🍁% life you wouldn't write code like this in Python; you would just use a
%🍁% dictionary!  So for the rest of this chapter, you have to imagine that
%🍁% dictionaries don't exist and you want to implement a data structure
%🍁% that maps from keys to values.  The operations you have to
%🍁% implement are:

我们使用 Python 来演示这些实现案例, 但在现实情况下,
你不需要用 Python 亲自写这样的代码, 只需用内建的字典对象即可!
因此以下的内容,是基于不存在我们需要的字典对象的假设,
我们想实现这样一个数据结构,将关键字映射为值。 你需要实现如下的函数运算:

%🍁% \begin{description}
%🍁%
%🍁% \item[{\tt add(k, v)}:] Add a new item that maps from key {\tt k}
%🍁% to value {\tt v}.  With a Python dictionary, {\tt d}, this operation
%🍁% is written {\tt d[k] = v}.
%🍁%
%🍁% \item[{\tt get(k)}:] Look up and return the value that corresponds
%🍁% to key {\tt k}.  With a Python dictionary, {\tt d}, this operation
%🍁% is written {\tt d[k]} or {\tt d.get(k)}.
%🍁%
%🍁% \end{description}

\begin{description}

\item [{\tt add(k, v)}:] 增加一个新的项,其从键 \li{k} 映射到值 \li{v}。
使用 Python 的字典 \li{d},该运算被写作 \li{d[k] = v}。

\item [{\tt get(k)}:] 查找并返回相应键的值。
如果使用 Python 的字典 \li{d} ,该运算被写作 \li{d[target]} 或 \li{d.get(target)}。

\end{description}

%🍁% For now, I assume that each key only appears once.
%🍁% The simplest implementation of this interface uses a list of
%🍁% tuples, where each tuple is a key-value pair.

现在,假设每个关键字只出现一次。
该接口最简单的实现是使用一个元组列表, 其中每个元组是关键字-值对。
\index{LinearMap@{\tt LinearMap}}

\begin{lstlisting}
class LinearMap:

    def __init__(self):
        self.items = []

    def add(self, k, v):
        self.items.append((k, v))

    def get(self, k):
        for key, val in self.items:
            if key == k:
                return val
        raise KeyError
\end{lstlisting}

%🍁% {\tt add} appends a key-value tuple to the list of items, which
%🍁% takes constant time.

\li{add} 向项列表追加一个关键字---值元组,这是常数时间。

%🍁% {\tt get} uses a {\tt for} loop to search the list:
%🍁% if it finds the target key it returns the corresponding value;
%🍁% otherwise it raises a {\tt KeyError}.
%🍁% So {\tt get} is linear.

\index{KeyError@{\tt KeyError}}

\li{get} 使用 \li{for} 循环搜索该列表: 如果它找到目标关键字,返回相应的值;
否则触发一个KeyError。因此get是线性的。

%🍁% An alternative is to keep the list sorted by key.  Then {\tt get}
%🍁% could use a bisection search, which is $O(\log n)$.  But inserting a
%🍁% new item in the middle of a list is linear, so this might not be the
%🍁% best option.  There are other data structures that can implement {\tt
%🍁%   add} and {\tt get} in log time, but that's still not as good as
%🍁% constant time, so let's move on.

另一个方案是保持列表按关键字排序。
那么 \li{get} 可以使用二分搜索,其是 $O(\log n)$ 。
但是在列表中间插入一个新的项是线性的, 因此这可能不是最好的选择。
有其它的数据结构(见:\href{http://en.wikipedia.org/wiki/Red-black_tree}{维基百科})
能在 对数级 时间内实现 \li{add} 和 \li{get},但是这仍然不如常数时间好,
那么我们继续。
\index{red-black tree}

\index{维基百科}


%🍁% One way to improve {\tt LinearMap} is to break the list of key-value
%🍁% pairs into smaller lists.  Here's an implementation called
%🍁% {\tt BetterMap}, which is a list of 100 LinearMaps.  As we'll see
%🍁% in a second, the order of growth for {\tt get} is still linear,
%🍁% but {\tt BetterMap} is a step on the path toward hashtables:

\index{BetterMap@{\tt BetterMap}}

另一种改良 \li{LinearMap} 的方法是 将 关键字-值 对的列表分成小列表。
这是一个被称作 \li{BetterMap} 的更好的实现,它是 100 个 \li{LinearMaps} 的列表。
正如一会儿我们将看到的,\li{get} 的增长量级仍然是线性的,
但是 \li{BetterMap} 是迈向哈希表的一步。

\begin{lstlisting}
class BetterMap:

    def __init__(self, n=100):
        self.maps = []
        for i in range(n):
            self.maps.append(LinearMap())

    def find_map(self, k):
        index = hash(k) % len(self.maps)
        return self.maps[index]

    def add(self, k, v):
        m = self.find_map(k)
        m.add(k, v)

    def get(self, k):
        m = self.find_map(k)
        return m.get(k)
\end{lstlisting}

%🍁% \verb"__init__" makes a list of {\tt n} {\tt LinearMap}s.

\li{__init__} 会生成一个由 \li{n} 个 \li{LinearMap} 组成的列表。

%🍁% \verb"find_map" is used by
%🍁% {\tt add} and {\tt get}
%🍁% to figure out which map to put the
%🍁% new item in, or which map to search.

\li{add} 和 \li{get} 使用 \li{find_map} 查找往哪一个列表中添加新项,或者对哪个列表进行检索。


%🍁% \verb"find_map" uses the built-in function {\tt hash}, which takes
%🍁% almost any Python object and returns an integer.  A limitation of this
%🍁% implementation is that it only works with hashable keys.  Mutable
%🍁% types like lists and dictionaries are unhashable.

\li{find_map} 使用内建 \li{hash} 函数,其接受几乎任何 Python 对象并返回一个整数。
这一实现的一个限制是它仅适用于哈希表关键字。 如列表和字典等易变的类型是不能哈希的。
\index{hash function}

%🍁% Hashable objects that are considered equivalent return the same hash
%🍁% value, but the converse is not necessarily true: two objects with
%🍁% different values can return the same hash value.

被认为是相等的可哈希的对象返回相同的哈希值,
但是反之不必成立:两个不同的对象能够返回相同的哈希值。

%🍁% \verb"find_map" uses the modulus operator to wrap the hash values
%🍁% into the range from 0 to {\tt len(self.maps)}, so the result is a legal
%🍁% index into the list.  Of course, this means that many different
%🍁% hash values will wrap onto the same index.  But if the hash function
%🍁% spreads things out pretty evenly (which is what hash functions
%🍁% are designed to do), then we expect $n/100$ items per LinearMap.

\li{find_map} 使用求余运算符将哈希值包在 0 到 \li{len(self.maps)} 之间,
因此结果是对于该列表合法的索引值。 当然,这意味着许多不同的哈希值将被包成相同的索引值。 但是如果哈希函数散布相当均匀(这是哈希函数被设计的初衷), 那么我们预计每个 \li{LinearMap} 有 $n/100$ 项。

%🍁% Since the run time of {\tt LinearMap.get} is proportional to the
%🍁% number of items, we expect BetterMap to be about 100 times faster
%🍁% than LinearMap.  The order of growth is still linear, but the
%🍁% leading coefficient is smaller.  That's nice, but still not
%🍁% as good as a hashtable.

由于 \li{LinearMap.get} 的运行时间与项数成正比,
我们预计 \li{BetterMap} 比 \li{LinearMap} 快100倍。
增长量级仍然是线性的,但是首系数变小了。 这很好,但是仍然不如哈希表好。

%🍁% Here (finally) is the crucial idea that makes hashtables fast: if you
%🍁% can keep the maximum length of the LinearMaps bounded, {\tt
%🍁%   LinearMap.get} is constant time.  All you have to do is keep track
%🍁% of the number of items and when the number of
%🍁% items per LinearMap exceeds a threshold, resize the hashtable by
%🍁% adding more LinearMaps.

下面是使哈希表变快的关键:
如果你能保证 \li{LinearMaps} 的最大长度是有上限的,则 \li{LinearMap.get} 的增长量级是常数时间。
你只需要跟踪项数并且当每个 \li{LinearMap} 的项数超过一个阈值时,
通过增加更多的 \li{LinearMaps} 调整哈希表的大小。
\index{bounded}

%🍁% Here is an implementation of a hashtable:

这是哈希表的一个实现:
\index{HashMap}

\begin{lstlisting}
class HashMap:

    def __init__(self):
        self.maps = BetterMap(2)
        self.num = 0

    def get(self, k):
        return self.maps.get(k)

    def add(self, k, v):
        if self.num == len(self.maps.maps):
            self.resize()

        self.maps.add(k, v)
        self.num += 1

    def resize(self):
        new_maps = BetterMap(self.num * 2)

        for m in self.maps.maps:
            for k, v in m.items:
                new_maps.add(k, v)

        self.maps = new_maps
\end{lstlisting}

%🍁% Each {\tt HashMap} contains a {\tt BetterMap}; \verb"__init__" starts
%🍁% with just 2 LinearMaps and initializes {\tt num}, which keeps track of
%🍁% the number of items.

每个 \li{HashMap} 包含一个 \li{BetterMap}。
\li{__init__} 开始仅有两个 \li{LinearMaps}, 并且初始化 \li{num}, 用于跟踪项的数目。

%🍁% {\tt get} just dispatches to {\tt BetterMap}.  The real work happens
%🍁% in {\tt add}, which checks the number of items and the size of the
%🍁% {\tt BetterMap}: if they are equal, the average number of items per
%🍁% LinearMap is 1, so it calls {\tt resize}.

\li{get} 仅仅用来调度 \li{BetterMap}。
真正的操作发生于 \li{add} 内,其检查项的数量以及 \li{BetterMap} 的大小:
如果它们相同,每个 \li{LinearMap} 的平均项数为 1,因此它调用 \li{resize}。

%🍁% {\tt resize} make a new {\tt BetterMap}, twice as big as the previous
%🍁% one, and then ``rehashes'' the items from the old map to the new.

\li{resize} 生成一个新的 \li{BetterMap},是之前的两倍大,
然后将像从旧表\li{map} ``重新哈希'' 至新的表。

%🍁% Rehashing is necessary because changing the number of LinearMaps
%🍁% changes the denominator of the modulus operator in
%🍁% \verb"find_map".  That means that some objects that used
%🍁% to hash into the same LinearMap will get split up (which is
%🍁% what we wanted, right?).

重哈希是很必要的,因为改变 \li{LinearMaps} 的数目也改变了 \li{find_map} 中求余运算的分母。
那意味着一些被包进相同的 \li{LinearMap} 的对象将被分离(这正是我们希望的,对吧?)
\index{rehashing}

%🍁% Rehashing is linear, so
%🍁% {\tt resize} is linear, which might seem bad, since I promised
%🍁% that {\tt add} would be constant time.  But remember that
%🍁% we don't have to resize every time, so {\tt add} is usually
%🍁% constant time and only occasionally linear.  The total amount
%🍁% of work to run {\tt add} $n$ times is proportional to $n$,
%🍁% so the average time of each {\tt add} is constant time!

重哈希是线性的,因此 \li{resize} 是线性的,
这可能看起来很糟糕,因为我保证 \li{add} 会是常数时间。
但是记住,我们不必没每次都调整,因此 \li{add} 通常是常数时间
只是偶然是线性的。  运行 \li{add} $n$ 次的整体操作量与 $n$ 成正比,
因此 \li{add} 的平均运行时间是常数时间!
\index{constant time}

%🍁% To see how this works, think about starting with an empty
%🍁% HashTable and adding a sequence of items.  We start with 2 LinearMaps,
%🍁% so the first 2 adds are fast (no resizing required).  Let's
%🍁% say that they take one unit of work each.  The next add
%🍁% requires a resize, so we have to rehash the first two
%🍁% items (let's call that 2 more units of work) and then
%🍁% add the third item (one more unit).  Adding the next item
%🍁% costs 1 unit, so the total so far is
%🍁% 6 units of work for 4 items.

为了弄清这是如何工作的,考虑以一个空的 \li{HashTable} 开始并增加一系列项。
我们以两个 \li{LinearMap} 开始,因此前两个 \li{add} 很快(不需要调整大小 \li{resize})。
我们假设它们每个操作花费一个工作单元。
下一个 \li{add} 需要进行一次 \li{resize},因此我们必须重哈希前两项
(我们将其算作调用两个额外的工作单元),然后增加第 3 项(又一个额外单元)。
增加下一项花费 1 个单元,所以目前为止添加4个项总共需要6个单元。

%🍁% The next {\tt add} costs 5 units, but the next three
%🍁% are only one unit each, so the total is 14 units for the
%🍁% first 8 adds.

下一个 \li{add} 花费5个单元,但是之后的3个操作每个只花费 1 个单元,
所以前 8 个 \li{add} 总共需要 14 个单元。

%🍁% The next {\tt add} costs 9 units, but then we can add 7 more
%🍁% before the next resize, so the total is 30 units for the
%🍁% first 16 adds.

下一个 \li{add} 花费9个单元,但是之后在下一次 调整大小 (\li{resize})
之前,可以再增加额外的7个,所以前16个add总共是30个单元。

%🍁% After 32 adds, the total cost is 62 units, and I hope you are starting
%🍁% to see a pattern.  After $n$ adds, where $n$ is a power of two, the
%🍁% total cost is $2n-2$ units, so the average work per add is
%🍁% a little less than 2 units.  When $n$ is a power of two, that's
%🍁% the best case; for other values of $n$ the average work is a little
%🍁% higher, but that's not important.  The important thing is that it
%🍁% is $O(1)$.

在 32 次 \li{add} 后,总共花费 62 个单元,我想你开始看到一个规律。
$n$ 次 add 后,其中 $n$ 是 2 的指数,总共花费是 $2n-2$ 个单元,
所以平均每个 \li{add} 操作的花费稍微少于 2 个单元。
当 $n$ 是 2 的指数时是最好的情况。
对于其它的 $n$ 值,平均花费稍高一点,但是那并不重要。
重要的是其增长量级为 $O(1)$ 。
\index{average cost}

%🍁% Figure~\ref{fig.hash} shows how this works graphically.  Each
%🍁% block represents a unit of work.  The columns show the total
%🍁% work for each add in order from left to right: the first two
%🍁% {\tt adds} cost 1 units, the third costs 3 units, etc.

图~\ref{fig.hash}展示这如何工作的。
每个块代表一个工作单元。按从左到右的顺序,每列显示每个add所需的单元:
前两个adds花费1个单元,第3个花费3个单元等等。

\begin{figure}
\centerline{\includegraphics[width=5.5in]{../source/figs/towers.pdf}}
% \caption{The cost of a hashtable add.\label{fig.hash}}
\caption{ \li{add} 操作在哈希表中的成本。 \label{fig.hash}}
\end{figure}

%🍁% The extra work of rehashing appears as a sequence of increasingly
%🍁% tall towers with increasing space between them.  Now if you knock
%🍁% over the towers, spreading the cost of resizing over all
%🍁% adds, you can see graphically that the total cost after $n$
%🍁% adds is $2n - 2$.

重新哈希的额外工作,表现为一系列不断增高的高塔,各自之间的距离越来越大。
现在,如果你打翻这些塔,将 调整大小 (\li{resize}) 的代价均摊到所有的 \li{add}操作上, 你会从图上看到 $n$ 个 \li{add} 的整体花费是 $2n - 2$ 。

%🍁% An important feature of this algorithm is that when we resize the
%🍁% HashTable it grows geometrically; that is, we multiply the size by a
%🍁% constant.  If you increase the size
%🍁% arithmetically---adding a fixed number each time---the average time
%🍁% per {\tt add} is linear.

该算法一个重要的特征是当我们 调整 (\li{resize}) 哈希表 (\li{HashTable})
大小的时候,它呈几何级增长。  也就是说,我们用常数乘以表的大小。
如果你按算术级增加大小 —— 每次增加固定的数目 —— 每个 \li{add} 操作的平均时间是线性的。
\index{geometric resizing}

%🍁% You can download my implementation of HashMap from
%🍁% \url{http://thinkpython2.com/code/Map.py}, but remember that there
%🍁% is no reason to use it; if you want a map, just use a Python dictionary.

你可以 \href{http://thinkpython2.com/code/Map.py}{在此}下载 到 \li{HashMap} 的实现代码, 但在实际环境中直接使用 Python 的{\em 字典} 即可 。

%🍁% \section{Glossary  |  术语表}
\section{术语表}

\begin{description}

%🍁% \item[analysis of algorithms:] A way to compare algorithms in terms of
%🍁% their run time and/or space requirements.

\item[算法分析 (analysis of algorithms)] 比较不同算法间运行时间和资源占用的分析方法。
\index{analysis of algorithms}

%🍁% \item[machine model:] A simplified representation of a computer used
%🍁% to describe algorithms.

\item[模型机器 (machine model)] 用于描述算法(性能)的简化的计算机表示。
\index{machine model}

%🍁% \item[worst case:] The input that makes a given algorithm run slowest (or
%🍁% require the most space.

\item[最坏情况 (worest case)] 对给定算法话最长时间运行(或占用做多资源)的输入。
\index{worst case}

%🍁% \item[leading term:] In a polynomial, the term with the highest exponent.

\item[首项 (leading term)] 在多项式中,拥有指数最高的项。
\index{leading term}

%🍁% \item[crossover point:] The problem size where two algorithms require
%🍁% the same run time or space.

\item[交叉点 (crossover point)] 对给定问题的某个规模,使得两个算法的求解需要相同运行时间或资源开销。
\index{crossover point}

%🍁% \item[order of growth:] A set of functions that all grow in a way
%🍁% considered equivalent for purposes of analysis of algorithms.
%🍁% For example, all functions that grow linearly belong to the same
%🍁% order of growth.

\item [增长量级 (order of growth)] 一个函数的集合,从算法分析的角度来看其中的函数的增长视为等价的。 例如, 线性递增的所有的函数都 属于 同一个增长量级。
\index{order of growth}

%🍁% \item[Big-Oh notation:] Notation for representing an order of growth;
%🍁% for example, $O(n)$ represents the set of functions that grow
%🍁% linearly.

\item[大O符号 (Big-Oh notation)] 代表增长量级的符号;例如, $O(n)$ 代表线性增长的函数集合。
\index{Big-Oh notation}

%🍁% \item[linear:] An algorithm whose run time is proportional to
%🍁% problem size, at least for large problem sizes.

\item[线性级 (linear)] 算法的运行时间和所求解问题的规模成正比, 至少是在问题规模较大时显现。
\index{linear}

%🍁% \item[quadratic:] An algorithm whose run time is proportional to
%🍁% $n^2$, where $n$ is a measure of problem size.

\item[二次方级 (quadratic)] 算法的运行时间 和 求解问题的规模的二次方($n^2$)成正比,$n$ 用于描述问题的规模。
\index{quadratic}

%🍁% \item[search:] The problem of locating an element of a collection
%🍁% (like a list or dictionary) or determining that it is not present.

\item[搜索 (search)] 在一个集合(例如列表或字典)中定位某个元素位置的问题,也等价于判断该元素是否存在于集合中。
\index{search}

%🍁% \item[hashtable:] A data structure that represents a collection of
%🍁% key-value pairs and performs search in constant time.

\item[哈希表 (hashtable)] 代表键-值对集合的一种数据结构,执行搜索操作只需常数时间。
\index{hashtable}

\end{description}
