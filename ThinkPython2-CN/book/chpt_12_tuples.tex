

%🍁% \chapter{Tuples | 元组}
\chapter{元组}
\label{tuplechap}

%🍁% This chapter presents one more built-in type, the tuple, and then
%🍁% shows how lists, dictionaries, and tuples work together.
%🍁% I also present a useful feature for variable-length argument lists,
%🍁% the gather and scatter operators.
%🍁%
%🍁% One note: there is no consensus on how to pronounce ``tuple''.
%🍁% Some people say ``tuh-ple'', which rhymes with ``supple''.  But
%🍁% in the context of programming, most people say ``too-ple'', which
%🍁% rhymes with ``quadruple''.

本章介绍另一个内建的类型 --- 元组\footnote{值得注意的是, ``tuple''并没有统一的发音, 有些人读``tuh-ple'', 音律类似于``supple'';而有人读``too-ple''音律类似于``quadruple''。  }, 同时说明如何结合使用列表、字典和元组。
后面的章节会介绍关于 可变长度参数列表 的有用功能, 以及\emph{汇集} 和 \emph{分散}操作。

%🍁% \section{Tuples are immutable | 元组是不可变的}
\section{元组是不可变的}
\index{tuple}  \index{type!tuple}  \index{sequence}

%🍁% A tuple is a sequence of values.  The values can be any type, and
%🍁% they are indexed by integers, so in that respect tuples are a lot
%🍁% like lists.  The important difference is that tuples are immutable.
\index{mutability}  \index{immutability}

元组是一组\emph{值}的序列。
其中的值可以是任意类型, 使用整数索引其位置, 因此元组与列表非常相似。
而重要的不同之处在于元组的不可变性。

%🍁% Syntactically, a tuple is a comma-separated list of values:

语法上,元组是用逗号隔开一系列值的列表:

\begin{lstlisting}
>>> t = 'a', 'b', 'c', 'd', 'e'
\end{lstlisting}
%
%🍁% Although it is not necessary, it is common to enclose tuples in
%🍁% parentheses:

虽然并非必须, 元组通常用括号括起来:

\index{parentheses!tuples in}

\begin{lstlisting}
>>> t = ('a', 'b', 'c', 'd', 'e')
\end{lstlisting}
%
%🍁% To create a tuple with a single element, you have to include a final
%🍁% comma:

使用单一元素建立元组时, 需要在结尾使用一个逗号:

\index{singleton}
\index{tuple!singleton}

\begin{lstlisting}
>>> t1 = 'a',
>>> type(t1)
<class 'tuple'>
\end{lstlisting}
%
%🍁% A value in parentheses is not a tuple:

将值放置在括号中并不会创建元组:

\begin{lstlisting}
>>> t2 = ('a')
>>> type(t2)
<class 'str'>
\end{lstlisting}
%
%🍁% Another way to create a tuple is the built-in function {\tt tuple}.
%🍁% With no argument, it creates an empty tuple:
\index{tuple function}
\index{function!tuple}

另一个建立元组的方法是使用内建函数 \li{tuple}。
在没有参数传递时它会产生一个空元组。

\begin{lstlisting}
>>> t = tuple()
>>> t
()
\end{lstlisting}

%
%🍁% If the argument is a sequence (string, list or tuple), the result
%🍁% is a tuple with the elements of the sequence:

如果实参是一个序列(字符串、列表或者元组), 结果将是包含序列内元素的一个元组。

\begin{lstlisting}
>>> t = tuple('lupins')
>>> t
('l', 'u', 'p', 'i', 'n', 's')
\end{lstlisting}
%
%🍁% Because {\tt tuple} is the name of a built-in function, you should
%🍁% avoid using it as a variable name.

因为 \li{tuple} 是内建函数名, 所以应该避免将它用于变量名。


%🍁% Most list operators also work on tuples.  The bracket operator
%🍁% indexes an element:

列表的大多数操作同样也适用于元组。  方括号运算符将索引一个元素:

\index{bracket operator}
\index{operator!bracket}

\begin{lstlisting}
>>> t = ('a', 'b', 'c', 'd', 'e')
>>> t[0]
'a'
\end{lstlisting}
%
%🍁% And the slice operator selects a range of elements.

切片操作可以选取一个范围内的元素:
\index{slice operator}  \index{operator!slice}
\index{tuple!slice}  \index{slice!tuple}
\index{切片操作符}  \index{操作符!切片}
\index{元组!切片}  \index{切片!元组}

\begin{lstlisting}
>>> t[1:3]
('b', 'c')
\end{lstlisting}
%
%🍁% But if you try to modify one of the elements of the tuple, you get
%🍁% an error:

但是, 如果你试更改图元组中的一个元素, 会得到错误信息:

\index{exception!TypeError}  \index{TypeError}
\index{item assignment}  \index{assignment!item}

\begin{lstlisting}
>>> t[0] = 'A'
TypeError: object doesn't support item assignment
\end{lstlisting}

%
%🍁% Because tuples are immutable, you can't modify the elements.  But you
%🍁% can replace one tuple with another:

因为元组是不可变的, 您无法改变其中的元素。
但是可以使用其他元组替换现有元组:

\begin{lstlisting}
>>> t = ('A',) + t[1:]
>>> t
('A', 'b', 'c', 'd', 'e')
\end{lstlisting}
%
%🍁% This statement makes a new tuple and then makes {\tt t} refer to it.

这个语句创建了一个新元组, 然后让 \li{t} 引用该元组。

%🍁% The relational operators work with tuples and other sequences;
%🍁% Python starts by comparing the first element from each
%🍁% sequence.  If they are equal, it goes on to the next elements,
%🍁% and so on, until it finds elements that differ.  Subsequent
%🍁% elements are not considered (even if they are really big).

关系型操作也适用于元组和其他序列;
Python 会首先比较序列中的第一个元素, 如果它们相等, 就继续比较下一组元素,
以此类推, 直至比值不同。
其后的元素(即便是差异很大)也不会再参与比较。

\index{comparison!tuple}
\index{tuple!comparison}

\begin{lstlisting}
>>> (0, 1, 2) < (0, 3, 4)
True
>>> (0, 1, 2000000) < (0, 3, 4)
True
\end{lstlisting}


%🍁% \section{Tuple assignment | 元组赋值}
\section{元组赋值}
\label{tuple.assignment} \index{tuple!assignment} \index{assignment!tuple}
\index{swap pattern} \index{pattern!swap}

%🍁% It is often useful to swap the values of two variables.
%🍁% With conventional assignments, you have to use a temporary
%🍁% variable.  For example, to swap {\tt a} and {\tt b}:

两个变量互换值的操作通常很有用。
按照传统的赋值方法, 你需要使用一个临时变量。
例如为了交换 \li{a}和 \li{b} 的值:

\begin{lstlisting}
>>> temp = a
>>> a = b
>>> b = temp
\end{lstlisting}
%
%🍁% This solution is cumbersome; {\bf tuple assignment} is more elegant:

这个方法很繁琐;通过{\bf 元组赋值}来实现更为优雅:

\begin{lstlisting}
>>> a, b = b, a
\end{lstlisting}
%
%🍁% The left side is a tuple of variables; the right side is a tuple of
%🍁% expressions.  Each value is assigned to its respective variable.
%🍁% All the expressions on the right side are evaluated before any
%🍁% of the assignments.

等号左侧是变量组成的元组;右侧是表达式组成的元组。
每个值都被赋给了对应的变量。
变量被重新赋值前, 将先对右侧的表达式进行求值。

%🍁% The number of variables on the left and the number of
%🍁% values on the right have to be the same:

使用元组赋值, 左右两侧变量数必须相同:

\index{exception!ValueError}  \index{ValueError}

\begin{lstlisting}
>>> a, b = 1, 2, 3
ValueError: too many values to unpack
\end{lstlisting}
%
%🍁% More generally, the right side can be any kind of sequence
%🍁% (string, list or tuple).  For example, to split an email address
%🍁% into a user name and a domain, you could write:

一般说来, 元组赋值时右侧表达式可以是任意类型(字符串、列表或者元组)的序列。  例如, 将一个电子邮箱地址分成用户名和域名, 你可以:

\index{split method}  \index{method!split}
\index{email address}

\begin{lstlisting}
>>> addr = 'monty@python.org'
>>> uname, domain = addr.split('@')
\end{lstlisting}

%
%🍁% The return value from {\tt split} is a list with two elements;
%🍁% the first element is assigned to {\tt uname}, the second to
%🍁% {\tt domain}.

 \li{split}函数返回的对象是一个包含两个元素的列表;第一个元素被赋给了 \li{uname}的变量, 第二个被赋给了 \li{domain}。

\begin{lstlisting}
>>> uname
'monty'
>>> domain
'python.org'
\end{lstlisting}
%

%🍁% \section{Tuples as return values | 元组作为返回值}
\section{元组作为返回值}
\index{tuple} \index{value!tuple} \index{return value!tuple}
\index{function, tuple as return value}

%🍁% Strictly speaking, a function can only return one value, but
%🍁% if the value is a tuple, the effect is the same as returning
%🍁% multiple values.  For example, if you want to divide two integers
%🍁% and compute the quotient and remainder, it is inefficient to
%🍁% compute {\tt x/y} and then {\tt x\%y}.  It is better to compute
%🍁% them both at the same time.

严格地说, 一个函数只能返回一个值, 但是如果这个返回值是元组, 其效果等同于返回多个值。  例如, 你想对两个整数做除法, 计算出商和余数, 依次计算出 \li{x/y}和 \li{x%y}是很低效的。
同时计算出这两个值更好。
\index{divmod}

%🍁% The built-in function {\tt divmod} takes two arguments and
%🍁% returns a tuple of two values, the quotient and remainder.
%🍁% You can store the result as a tuple:

内建函数\href{https://docs.python.org/3/library/functions.html#divmod}{ \li{divmod}}接受两个参数, 返回包含两个值的元组 --- 商和余数。
可以使用元组来存储返回值:

\begin{lstlisting}
>>> t = divmod(7, 3)
>>> t
(2, 1)
\end{lstlisting}

%
%🍁% Or use tuple assignment to store the elements separately:

或者使用元组赋值分别存储它们:

\index{tuple assignment}  \index{assignment!tuple}

\begin{lstlisting}
>>> quot, rem = divmod(7, 3)
>>> quot
2
>>> rem
1
\end{lstlisting}

%
%🍁% Here is an example of a function that returns a tuple:

下面是另一个返回元组作为结果的函数例子:

\begin{lstlisting}
def min_max(t):
    return min(t), max(t)
\end{lstlisting}

%
%🍁% {\tt max} and {\tt min} are built-in functions that find
%🍁% the largest and smallest elements of a sequence.  \verb"min_max"
%🍁% computes both and returns a tuple of two values.

 \li{max} 和  \li{min} 是用于找出一组元素序列中最大值和最小值的内建函数, \li{min_max}函数同时计算出这两个值, 并返回二者组成的元组。
\index{max function} \index{function!max}
\index{min function} \index{function!min}


%🍁% \section{Variable-length argument tuples | 可变长度参数元组}
\section{可变长度参数元组}
\label{gather}
\index{variable-length argument tuple} \index{argument!variable-length tuple}
\index{gather} \index{parameter!gather} \index{argument!gather}

%🍁% Functions can take a variable number of arguments.  A parameter
%🍁% name that begins with {\tt *} {\bf gathers} arguments into
%🍁% a tuple.  For example, {\tt printall}
%🍁% takes any number of arguments and prints them:

函数可以接受可变数量的参数。  以 {\bf *} 开头的形参将输入的参数 \emph{汇集} 到一个元组中。
例如, \li{printall} 可以接受任意数量的参数, 并将它们打印出来:

\begin{lstlisting}
def printall(*args):
    print(args)
\end{lstlisting}

%
%🍁% The gather parameter can have any name you like, but {\tt args} is
%🍁% conventional.  Here's how the function works:

汇集的形参可以使用任意名字, 但是习惯使用 \li{args}。
以下是这个函数的调用效果:

\begin{lstlisting}
>>> printall(1, 2.0, '3')
(1, 2.0, '3')
\end{lstlisting}

%
%🍁% The complement of gather is {\bf scatter}.  If you have a
%🍁% sequence of values and you want to pass it to a function
%🍁% as multiple arguments, you can use the {\tt *} operator.
%🍁% For example, {\tt divmod} takes exactly two arguments; it
%🍁% doesn't work with a tuple:

与汇集相对的是\emph{分散}{\bf scatter}。
如果你有一个值的序列, 并且希望将其作为多个参数传递给一个函数,
你可以使用运算符 \li{*}。
例如, \li{divmod} 需要接受两个实参;一个元组则无法作为参数传递进去:

\index{scatter} \index{argument scatter} \index{TypeError}
\index{exception!TypeError}

\begin{lstlisting}
>>> t = (7, 3)
>>> divmod(t)
TypeError: divmod expected 2 arguments, got 1
\end{lstlisting}

%
%🍁% But if you scatter the tuple, it works:

但是如果将这个元组分散, 它就可以被传递进函数:

\begin{lstlisting}
>>> divmod(*t)
(2, 1)
\end{lstlisting}

%
%🍁% Many of the built-in functions use
%🍁% variable-length argument tuples.  For example, {\tt max}
%🍁% and {\tt min} can take any number of arguments:

多数内建函数使用可变长度参数元组。
例如, \li{max} 和  \li{min} 可以接受任意数量的实参。

\index{max function} \index{function!max}
\index{min function} \index{function!min}

\begin{lstlisting}
>>> max(1, 2, 3)
3
\end{lstlisting}

%
%🍁% But {\tt sum} does not.

但是求和操作 \li{sum} 不行:
\index{sum function} \index{function!sum}

\begin{lstlisting}
>>> sum(1, 2, 3)
TypeError: sum expected at most 2 arguments, got 3
\end{lstlisting}

%
%🍁% As an exercise, write a function called {\tt sumall} that takes any number
%🍁% of arguments and returns their sum.

我们尝试编写一个叫做  \li{sumall}的函数作为练习,
使它能够接受任何数量的传参并返回它们的和。


%🍁% \section{Lists and tuples | 列表和元组}
\section{列表和元组}
\index{zip function} \index{function!zip}

%🍁% {\tt zip} is a built-in function that takes two or more sequences and
%🍁% returns a list of tuples where each tuple contains one
%🍁% element from each sequence.  The name of the function refers to
%🍁% a zipper, which joins and interleaves two rows of teeth.

 \li{zip} 是一个内建函数, 可以接受将两个或多个序列组, 并返回一个元组列表,
其中每个元组包含了各个序列中相对位置的一个元素。
这个函数的名称来自名词拉链 (zipper), 后者将两片链齿连接拼合在一起。


%🍁% This example zips a string and a list:

下面的示例对一个字符串和列表使用 \li{zip} 函数:

\begin{lstlisting}
>>> s = 'abc'
>>> t = [0, 1, 2]
>>> zip(s, t)
<zip object at 0x7f7d0a9e7c48>
\end{lstlisting}

%
%🍁% The result is a {\bf zip object} that knows how to iterate through
%🍁% the pairs.  The most common use of {\tt zip} is in a {\tt for} loop:

输出的结果是一个 {\em \li{zip} 对象}, 包含了如何对其中元素进行迭代的信息。
\li{zip} 函数最常用于 \li{for} 循环:

\begin{lstlisting}
>>> for pair in zip(s, t):
...     print(pair)
...
('a', 0)
('b', 1)
('c', 2)
\end{lstlisting}

%
%🍁% A zip object is a kind of {\bf iterator}, which is any object
%🍁% that iterates through a sequence.  Iterators are similar to lists in some
%🍁% ways, but unlike lists, you can't use an index to select an element from
%🍁% an iterator.

\href{https://docs.python.org/3/library/functions.html#zip}{\li{zip}}对象是一个友善的 {\bf 迭代器}, 是指任何一种能够按照某个序列迭代的对象。  迭代器在某些方面与列表非常相似, 不同之处在于, 你无法通过索引来选择迭代器中的某个元素。
\index{iterator} \index{迭代器}

%🍁% If you want to use list operators and methods, you can
%🍁% use a zip object to make a list:

如果你想使用列表操作符和方法, 你可以通过 \li {zip}对象创造一个列表:

\begin{lstlisting}
>>> list(zip(s, t))
[('a', 0), ('b', 1), ('c', 2)]
\end{lstlisting}

%
%🍁% The result is a list of tuples; in this example, each tuple contains
%🍁% a character from the string and the corresponding element from
%🍁% the list.

结果就是一个包含若干元组的列表;在这个例子中, 每个元组又包含了字符串中的一个字符和列表 \li {t} 中对应的一个元素。
\index{list!of tuples}

%🍁% If the sequences are not the same length, the result has the
%🍁% length of the shorter one.

如果用于创建的序列长度不一, 返回的对象的长度以最短序列的长度为准。

\begin{lstlisting}
>>> list(zip('Anne', 'Elk'))
[('A', 'E'), ('n', 'l'), ('n', 'k')]
\end{lstlisting}

%
%🍁% You can use tuple assignment in a {\tt for} loop to traverse a list of
%🍁% tuples:

你可以在 \li{for} 循环中使用元组赋值, 遍历包含元组的列表:

\index{traversal} \index{tuple assignment} \index{assignment!tuple}

\begin{lstlisting}
t = [('a', 0), ('b', 1), ('c', 2)]
for letter, number in t:
    print(number, letter)
\end{lstlisting}

%
%🍁% Each time through the loop, Python selects the next tuple in
%🍁% the list and assigns the elements to {\tt letter} and
%🍁% {\tt number}.  The output of this loop is:

循环中的每次执行, Python 会选择列表中的下一个元组,
并将其内容赋给  \li{letter} 和  \li{number}。
因此循环打印的输出会是这样:
\index{loop}

\begin{lstlisting}
0 a
1 b
2 c
\end{lstlisting}

%
%🍁% If you combine {\tt zip}, {\tt for} and tuple assignment, you get a
%🍁% useful idiom for traversing two (or more) sequences at the same
%🍁% time.  For example, \verb"has_match" takes two sequences, {\tt t1} and
%🍁% {\tt t2}, and returns {\tt True} if there is an index {\tt i}
%🍁% such that {\tt t1[i] == t2[i]}:

如果将 \li{zip}、 \li{for}循环和元组赋值结合起来使用,
你会得到一个可以同时遍历两个(甚至多个)序列的惯用法。
例如, \li{has_match} 接受两个序列, \li{t1} 和 \li{t2},
如果存在索引满足 \li{t1[i] == t2[i]} \li {i}, 则返回 \li{True}:
\index{for loop}

\begin{lstlisting}
def has_match(t1, t2):
    for x, y in zip(t1, t2):
        if x == y:
            return True
    return False
\end{lstlisting}

%
%🍁% If you need to traverse the elements of a sequence and their
%🍁% indices, you can use the built-in function {\tt enumerate}:

如果需要遍历一个序列的元素以及它们的索引号, 你可以使用内建函数 \li{enumerate}:
\index{traversal} \index{enumerate function} \index{function!enumerate}

\begin{lstlisting}
for index, element in enumerate('abc'):
    print(index, element)
\end{lstlisting}

%
%🍁% The result from {\tt enumerate} is an enumerate object, which
%🍁% iterates a sequence of pairs; each pair contains an index (starting
%🍁% from 0) and an element from the given sequence.
%🍁% In this example, the output is

\li{enumerate} 的返回结果是一个 枚举对象(enumerate object),
它可基于一个包含若干个 \emph{对} 的序列进行迭代,
每个对包含了(从0开始计数)的索引号和给定序列中对应的元素。
在刚才的例子中, 对应的输出结果会和前例一样:

\begin{lstlisting}
0 a
1 b
2 c
\end{lstlisting}

%
%🍁% Again.
\index{iterator}    \index{object!enumerate}    \index{enumerate object}


%🍁% \section{Dictionaries and tuples | 字典和元组}
\section{字典和元组}
\label{dictuple}
\index{dictionary} \index{items method}
\index{method!items} \index{key-value pair}

%🍁% Dictionaries have a method called {\tt items} that returns a sequence of
%🍁% tuples, where each tuple is a key-value pair.

字典对象有一个内建方法叫做 \href{https://docs.python.org/3/library/stdtypes.html?highlight=items#dict.items}{ \li{itmes} }, 它返回由多个元组组成的序列, 其中每个元组是一个键值对。

\begin{lstlisting}
>>> d = {'a':0, 'b':1, 'c':2}
>>> t = d.items()
>>> t
dict_items([('c', 2), ('a', 0), ('b', 1)])
\end{lstlisting}

%
%🍁% The result is a \verb"dict_items" object, which is an iterator that
%🍁% iterates the key-value pairs.  You can use it in a {\tt for} loop
%🍁% like this:

其结果是一个 \li{dict_itmes} 对象, 这是一个对键值对进行迭代的迭代器。
你可以在 \li{for} 循环中像这样使用它:
\index{iterator}

\begin{lstlisting}
>>> for key, value in d.items():
...     print(key, value)
...
c 2
a 0
b 1
\end{lstlisting}

%
%🍁% As you should expect from a dictionary, the items are in no
%🍁% particular order.

由于是字典生成的对象, 你应该猜到了这些项是无序的。

%🍁% Going in the other direction, you can use a list of tuples to
%🍁% initialize a new dictionary:

另一方面, 你可以使用元组的列表初始化一个新的字典:
\index{dictionary!initialize}

\begin{lstlisting}
>>> t = [('a', 0), ('c', 2), ('b', 1)]
>>> d = dict(t)
>>> d
{'a': 0, 'c': 2, 'b': 1}
\end{lstlisting}

%🍁% Combining {\tt dict} with {\tt zip} yields a concise way
%🍁% to create a dictionary:

将 \li{dict} 和 \li{zip}结合使用, 可以很简洁地创建一个字典:
\index{zip function!use with dict}

\begin{lstlisting}
>>> d = dict(zip('abc', range(3)))
>>> d
{'a': 0, 'c': 2, 'b': 1}
\end{lstlisting}

%
%🍁% The dictionary method {\tt update} also takes a list of tuples
%🍁% and adds them, as key-value pairs, to an existing dictionary.


字典的 \li{update} 方法也接受元组的列表, 并作为键-值对把它们添加到已有的字典中。

\index{update method}  \index{method!update}
\index{traverse!dictionary}  \index{dictionary!traversal}

%🍁% It is common to use tuples as keys in dictionaries (primarily because
%🍁% you can't use lists).  For example, a telephone directory might map
%🍁% from last-name, first-name pairs to telephone numbers.  Assuming
%🍁% that we have defined {\tt last}, {\tt first} and {\tt number}, we
%🍁% could write:

在字典中使用元组作为键(主要因为无法使用列表)的做法很常见。
例如, 一个电话簿可能会基于用户的姓-名对, 来映射至号码。
假设我们已经定义了 \li{last} 、 \li{first} 和 \li{number} 三个变量,
我们可以这样实现映射:

\index{tuple!as key in dictionary}
\index{hashable}

\begin{lstlisting}
directory[last, first] = number
\end{lstlisting}

%
%🍁% The expression in brackets is a tuple.  We could use tuple
%🍁% assignment to traverse this dictionary.

方括号中的表达式是一个元组。
我们可以通过元组赋值来遍历这个字典:
\index{tuple!in brackets}

\begin{lstlisting}
for last, first in directory:
    print(first, last, directory[last,first])
\end{lstlisting}

%
%🍁% This loop traverses the keys in {\tt directory}, which are tuples.  It
%🍁% assigns the elements of each tuple to {\tt last} and {\tt first}, then
%🍁% prints the name and corresponding telephone number.

该循环遍历 电话簿 \li{directory}中的键, 它们其实是元组。
循环将元组的元素赋给 \li{last} 和 \li{first} , 然后打印出姓名和对应的电话号码。

%🍁% There are two ways to represent tuples in a state diagram.  The more
%🍁% detailed version shows the indices and elements just as they appear in
%🍁% a list.  For example, the tuple \verb"('Cleese', 'John')" would appear
%🍁% as in Figure~\ref{fig.tuple1}.

在状态图中有两种表示元组的方法。
更详细的版本是, 索引号和对应元素就像列表一样存放在元组中。
例如, 元组 \li{('Cleese', 'John')}可像图~\ref{fig.tuple1}中那样存放。
\index{state diagram} \index{diagram!state}

\begin{figure}
\centerline
{\includegraphics[scale=0.8]{../source/figs/tuple1.pdf}}
\caption{State diagram.}
\label{fig.tuple1}
\end{figure}

%🍁% But in a larger diagram you might want to leave out the
%🍁% details.  For example, a diagram of the telephone directory might
%🍁% appear as in Figure~\ref{fig.dict2}.

在更大的图中, 我们忽略这些细节。
该电话簿的状态图可能如图~\ref{fig.dict2}所示。

\begin{figure}
\centerline
{\includegraphics[scale=0.8]{../source/figs/dict2.pdf}}
\caption{State diagram.}
\label{fig.dict2}
\end{figure}

%🍁% Here the tuples are shown using Python syntax as a graphical
%🍁% shorthand.  The telephone number in the diagram is the complaints line
%🍁% for the BBC, so please don't call it.

因此, Python风格的元组用法可用这两幅图来描述。
此图中的电话号码是 BBC 的投诉热线, 请不要拨打它。


%🍁% \section{Sequences of sequences | 序列嵌套}
\section{序列嵌套}
\index{sequence}

%🍁% I have focused on lists of tuples, but almost all of the examples in
%🍁% this chapter also work with lists of lists, tuples of tuples, and
%🍁% tuples of lists.  To avoid enumerating the possible combinations, it
%🍁% is sometimes easier to talk about sequences of sequences.

我已经介绍了包含元组的列表, 事实上, 本章大多数例子也适用于列表嵌套列表、 元组嵌套元组, 以及元组嵌套列表。
为了避免 --- 穷举这类可能的嵌套组合, 我们简称为序列嵌套。

%🍁% In many contexts, the different kinds of sequences (strings, lists and
%🍁% tuples) can be used interchangeably.  So how should you choose one
%🍁% over the others?

在很多情况下, 不同类型的序列(字符串、列表、元组)可以互换使用。  因此, 我们如何选用合适的嵌套对象呢?
\index{string} \index{list} \index{tuple} \index{mutability}
\index{immutability}

%🍁% To start with the obvious, strings are more limited than other
%🍁% sequences because the elements have to be characters.  They are
%🍁% also immutable.  If you need the ability to change the characters
%🍁% in a string (as opposed to creating a new string), you might
%🍁% want to use a list of characters instead.

首先, 显而易见的是, 字符串比其他序列的限制更多, 因为它的所有元素都是字符, 且字符串不可变。
如果你希望能够改变字符在字符串中的位置, 使用列表嵌套字符比较合适。

%🍁% Lists are more common than tuples, mostly because they are mutable.
%🍁% But there are a few cases where you might prefer tuples:

列表比元组更常见, 这源于它们可变性的易用。
但是有些情况下, 你会更倾向于使用元组:

%🍁% \begin{enumerate}
%🍁%
%🍁% \item In some contexts, like a {\tt return} statement, it is
%🍁% syntactically simpler to create a tuple than a list.
%🍁%
%🍁% \item If you want to use a sequence as a dictionary key, you
%🍁% have to use an immutable type like a tuple or string.
%🍁%
%🍁% \item If you are passing a sequence as an argument to a function,
%🍁% using tuples reduces the potential for unexpected behavior
%🍁% due to aliasing.
%🍁%
%🍁% \end{enumerate}

\begin{enumerate}

\item 在一些情况下(例如 \li{return}语句), 从句式上生成一个元组比列表要简单。

\item 如果你想使用一个序列作为字典的键, 那么你必须使用元组或字符串这样的不可变类型。

\item 如果你向函数传入一个序列作为参数, 那么使用元组以降低由于别名而产生的意外行为的可能性。

\end{enumerate}

%🍁% Because tuples are immutable, they don't provide methods like {\tt
%🍁%   sort} and {\tt reverse}, which modify existing lists.  But Python
%🍁% provides the built-in function {\tt sorted}, which takes any sequence
%🍁% and returns a new list with the same elements in sorted order, and
%🍁% {\tt reversed}, which takes a sequence and returns an iterator that
%🍁% traverses the list in reverse order.

正由于元组的不可变性, 它们没有类似 \li{sort} 和 \li{reverser} 这样修改现有列表的方法。
然而 Python 提供了内建函数  \li{sorted}, 用于对任意序列排序并输出相同元素的列表, 以及  \li {reversed}, 用于对序列逆向排序并生成一个可以遍历的迭代器。

\index{sorted function} \index{function!sorted} \index{reversed function}
\index{function!reversed} \index{iterator}


%🍁% \section{Debugging  |  调试}
\section{调试}
\index{debugging} \index{data structure}
\index{shape error} \index{error!shape}

%🍁% Lists, dictionaries and tuples are examples of {\bf data
%🍁%   structures}; in this chapter we are starting to see compound data
%🍁% structures, like lists of tuples, or dictionaries that contain tuples
%🍁% as keys and lists as values.  Compound data structures are useful, but
%🍁% they are prone to what I call {\bf shape errors}; that is, errors
%🍁% caused when a data structure has the wrong type, size, or structure.
%🍁% For example, if you are expecting a list with one integer and I
%🍁% give you a plain old integer (not in a list), it won't work.

列表、  字典 和 元组 都是 {\em 数据结构} ({\bf data structures});
本章中, 我们开始接触到 复合数据结构 ({\bf compound data structures}),
如: 列表嵌套元组, 又如使用元组作为键而列表作为值的字典。
复合数据结构非常实用, 然而使用时容易出现所谓的 {\em 形状错误} ({\bf shape errors}), 也就是说由于数据结构的类型、大小或结构问题而引发的错误。
例如, 当你希望使用封装整数的列表时却用成了没被列表包含的一串整数。
\index{structshape module} \index{module!structshape}

%🍁% To help debug these kinds of errors, I have written a module
%🍁% called {\tt structshape} that provides a function, also called
%🍁% {\tt structshape}, that takes any kind of data structure as
%🍁% an argument and returns a string that summarizes its shape.
%🍁% You can download it from \url{http://thinkpython2.com/code/structshape.py}

为了方面调试这类错误, 我编写了一个叫做  \li{structshape} 的模块,
它提供了一个名为 \li{structshape} 的函数, 可以接受任意类型的数据结构作为实参, 然后返回一个描述它形状的字符串。
你可以在\href{http://thinkpython2.com/code/structshape.py}{这里}下载到它(\url{http://thinkpython2.com/code/structshape.py})。

%🍁% Here's the result for a simple list:

下面是用该模块调试一个简单列表的示例:

\begin{lstlisting}
>>> from structshape import structshape
>>> t = [1, 2, 3]
>>> structshape(t)
'list of 3 int'
\end{lstlisting}

%
%🍁% A fancier program might write ``list of 3 int{\em s}'', but it
%🍁% was easier not to deal with plurals.  Here's a list of lists:

更完美的程序应该显示 ``list of 3 int{\em s}'', 但是忽略英文复数使程序简单的多。
我们再看一个列表嵌套的例子:

\begin{lstlisting}
>>> t2 = [[1,2], [3,4], [5,6]]
>>> structshape(t2)
'list of 3 list of 2 int'
\end{lstlisting}

%
%🍁% If the elements of the list are not the same type,
%🍁% {\tt structshape} groups them, in order, by type:

如果列表内的元素不是相同类型, \li{structshape} 会按照类型的顺序进行分组:

\begin{lstlisting}
>>> t3 = [1, 2, 3, 4.0, '5', '6', [7], [8], 9]
>>> structshape(t3)
'list of (3 int, float, 2 str, 2 list of int, int)'
\end{lstlisting}

%
%🍁% Here's a list of tuples:

下面是一个元组列表的例子:

\begin{lstlisting}
>>> s = 'abc'
>>> lt = list(zip(t, s))
>>> structshape(lt)
'list of 3 tuple of (int, str)'
\end{lstlisting}

%
%🍁% And here's a dictionary with 3 items that map integers to strings.

下面是一个字典的例子, 其中包含三个将整数映射至字符串的项:

\begin{lstlisting}
>>> d = dict(lt)
>>> structshape(d)
'dict of 3 int->str'
\end{lstlisting}

%
%🍁% If you are having trouble keeping track of your data structures,
%🍁% {\tt structshape} can help.

如果你在追踪数据结构的类型上遇到了困难, 可以使用 \li{structshape} 来帮助分析。

%🍁% \section{Glossary  |  术语表}
\section{术语表}

\begin{description}

%🍁% \item[tuple:] An immutable sequence of elements.

\item[元组 (tuple):] 一组不可变的元素的序列。
\index{tuple}

%🍁% \item[tuple assignment:] An assignment with a sequence on the
%🍁% right side and a tuple of variables on the left.  The right
%🍁% side is evaluated and then its elements are assigned to the
%🍁% variables on the left.

\item[元组赋值 (tuple assignment):] 一种赋值方式, 通过等号右侧的序列向等号左侧的一组变量的元组进行赋值。
右侧的表达式先求值, 然后其元素被赋值给左侧元组中对应的变量。
\index{tuple assignment} \index{assignment!tuple}

%🍁% \item[gather:] The operation of assembling a variable-length
%🍁% argument tuple.
\index{gather}

\item[汇集 (gather):] 组装可变长度变量元组的一种操作。

%🍁% \item[scatter:] The operation of treating a sequence as a list of
%🍁% arguments.
\index{scatter}

\item[分散 (scatter):] 将一个序列变换成一个参数列表的操作。


%🍁% \item[zip object:] The result of calling a built-in function {\tt zip};
%🍁% an object that iterates through a sequence of tuples.

\item[zip 对象:] 使用内建函数 \li{zip} 所返回的结果; 它是一个可通过元组序列逐个迭代的对象。
\index{zip object} \index{object!zip}

%🍁% \item[iterator:] An object that can iterate through a sequence, but
%🍁% which does not provide list operators and methods.

\item[迭代器 (iterator):]: 一个可以对序列进行迭代的对象, 但是并不提供列表操作符和方法。
\index{iterator}

%🍁% \item[data structure:] A collection of related values, often
%🍁% organized in lists, dictionaries, tuples, etc.

\item[数据结构 (data structure):] 一个由关联值组成的数据集合, 通常组织成列表、 字典、 元组等。
\index{data structure}

%🍁% \item[shape error:] An error caused because a value has the
%🍁% wrong shape; that is, the wrong type or size.

\item[形状错误:] 由于某个值的形状出错, 而导致的错误; 即拥有错误的类型或大小。
\index{shape}

\end{description}


%🍁% \section{Exercises  |  练习}

\section{练习}

\begin{exercise}
%🍁% Write a function called \verb"most_frequent" that takes a string and
%🍁% prints the letters in decreasing order of frequency.  Find text
%🍁% samples from several different languages and see how letter frequency
%🍁% varies between languages.  Compare your results with the tables at
%🍁% \url{http://en.wikipedia.org/wiki/Letter_frequencies}.  Solution:
%🍁% \url{http://thinkpython2.com/code/most_frequent.py}.

编写一个名为 {\em  \li{most_frequent}} 的函数,
接受一个字符串, 并按字符出现的频率 降序打印字母。
找一些不同语言的文本样本, 来试试看不同语言之间字母频率的区别。
将你的结果和维基百科的
\href{http://en.wikipedia.org/wiki/Letter_frequencies}{字母频率} 进行比较。

\href{http://thinkpython2.com/code/most_frequent.py}{参考答案}

\index{letter frequency} \index{frequency!letter}
\index{字母频度} \index{频度!字母}

\index{维基百科}


\end{exercise}


\begin{exercise}
\label{anagrams}
\index{anagram set}  \index{set!anagram}

%🍁% More anagrams!
易位构词游戏 {\em (\href{https://zh.wikipedia.org/wiki/%E6%98%93%E4%BD%8D%E6%9E%84%E8%AF%8D%E6%B8%B8%E6%88%8F}{anagrams})}!

\begin{enumerate}

%🍁% \item Write a program
%🍁% that reads a word list from a file (see Section~\ref{wordlist}) and
%🍁% prints all the sets of words that are anagrams.
%🍁%
%🍁% Here is an example of what the output might look like:
%🍁%
%🍁% \begin{lstlisting}
%🍁% ['deltas', 'desalt', 'lasted', 'salted', 'slated', 'staled']
%🍁% ['retainers', 'ternaries']
%🍁% ['generating', 'greatening']
%🍁% ['resmelts', 'smelters', 'termless']
%🍁% \end{lstlisting}
%🍁%
%🍁% %
%🍁% Hint: you might want to build a dictionary that maps from a
%🍁% collection of letters to a list of words that can be spelled with those
%🍁% letters.  The question is, how can you represent the collection of
%🍁% letters in a way that can be used as a key?
%🍁%
%🍁% \item Modify the previous program so that it prints the longest list
%🍁% of anagrams first, followed by the second longest, and so on.
%🍁% \index{Scrabble}
%🍁% \index{bingo}
%🍁%
%🍁% \item In Scrabble a ``bingo'' is when you play all seven tiles in
%🍁% your rack, along with a letter on the board, to form an eight-letter
%🍁% word.  What collection of 8 letters forms the most possible bingos?
%🍁% Hint: there are seven.
%🍁%
%🍁% % (7, ['angriest', 'astringe', 'ganister', 'gantries', 'granites',
%🍁% % 'ingrates', 'rangiest'])
%🍁%
%🍁% Solution: \url{http://thinkpython2.com/code/anagram_sets.py}.



\item 编写一个程序, 使之能从文件中读取单词的列表 (参考章节~{\em \ref{wordlist}}) 并且打印出所有符合异位构词的组合。

下面是一个输出异位构词的样例:

{\em
\begin{lstlisting}
['deltas', 'desalt', 'lasted', 'salted', 'slated', 'staled']
['retainers', 'ternaries']
['generating', 'greatening']
['resmelts', 'smelters', 'termless']
\end{lstlisting}
}

提示:也许你可以建立一个字典, 用于映射一个字符集合到一个该集合可异位构词的词汇集合。

\item 改写前面的程序, 使之首先打印包含异位构词数量最多的词汇列表, 第二多次之, 依次按异位构词数量排列。

\item \href{https://en.wikipedia.org/wiki/Scrabble}{{\em Scrabble}} \href{https://zh.wikipedia.org/wiki/Scrabble}{拼字游戏} 中, 游戏胜利{\em (``bingo'')}指的是你利用手里的全部七个字母, 与图版上的那个字母一起构成一个 {\em 8} 个字母的单词。  哪八个字母能够达成最多的 {\em ``bingo''?} 提示:最多有7种胜利方式。

\href{http://thinkpython2.com/code/anagram_sets.py}{参考答案}

\end{enumerate}

\end{exercise}

\begin{exercise}
\index{metathesis}

%🍁% Two words form a ``metathesis pair'' if you can transform one into the
%🍁% other by swapping two letters; for example, ``converse'' and
%🍁% ``conserve''.  Write a program that finds all of the metathesis pairs
%🍁% in the dictionary.  Hint: don't test all pairs of words, and don't
%🍁% test all possible swaps.  Solution:
%🍁% \url{http://thinkpython2.com/code/metathesis.py}.  Credit: This
%🍁% exercise is inspired by an example at \url{http://puzzlers.org}.

如果两个单词中的某一单词可以通过调换两个字母变为另一个, 这两个单词就构成了
``换位对 {\em (metatheisi pair)}''; 比如 {\em ``converse''} 和 {\em ``conserve''}。
编写一个程序, 找出给定字典里所有的 ``换位对''。\footnote{提示: 不用测试所有的单词组合, 也不用测试所有的字母调换组合。  }

\href{http://thinkpython2.com/code/metathesis.py}{参考答案}

\footnote{这个练习受\href{http://puzzlers.org}{http://puzzlers.org}的案例启发而成。  }

\end{exercise}


\begin{exercise}
\index{Car Talk}
\index{Puzzler}

%🍁% Here's another Car Talk Puzzler
%🍁% (\url{http://www.cartalk.com/content/puzzlers}):
%🍁%
%🍁% \begin{quote}
%🍁% What is the longest English word, that remains a valid English word,
%🍁% as you remove its letters one at a time?
%🍁%
%🍁% Now, letters can be removed from either end, or the middle, but you
%🍁% can't rearrange any of the letters. Every time you drop a letter, you
%🍁% wind up with another English word. If you do that, you're eventually
%🍁% going to wind up with one letter and that too is going to be an
%🍁% English word---one that's found in the dictionary. I want to know
%🍁% what's the longest word and how many letters does it
%🍁% have?
%🍁%
%🍁% I'm going to give you a little modest example: Sprite. Ok? You start
%🍁% off with sprite, you take a letter off, one from the interior of the
%🍁% word, take the r away, and we're left with the word spite, then we
%🍁% take the e off the end, we're left with spit, we take the s off, we're
%🍁% left with pit, it, and I.
%🍁% \end{quote}

另一个来自 {\em Car Talk} 的字谜题 \href{http://www.cartalk.com/content/puzzlers}{{\em car talk puzzler}} :

\begin{quote}

如果你每一次从单词中删掉一个字母以后, 剩下的字符仍然能构成一个单词, 请问世界上符合条件的最长单词是什么?

注意, 被删掉的字母可以位于首尾或是中间, 但不允许重新去排列剩下的字母。
每次移除一个字母后 , 你会得到一个新单词。
这样一直下去, 最终你只剩一个字母, 并且它也是一个单词——可以在字典中查到。  我想知道, 符合条件的最长单词是什么?它由多少个字母构成?

我先给出一个短小的例子: {\em ``Sprite''}, 一开始是 {\em sprite} , 我们可以拿掉中间的 {\em `r'} 从而获得单词 {\em spite}, 拿去字母 {\em `e'} 得到 {\em spit}, 再去掉 {\em `s'} 剩下 {\em pit}, {\em it}, 最后 {\em I}。
\end{quote}

\index{reducible word} \index{word, reducible}

%🍁% Write a program to find all words that can be reduced in this way,
%🍁% and then find the longest one.
%🍁%
%🍁% This exercise is a little more challenging than most, so here are
%🍁% some suggestions:

%🍁% \begin{enumerate}
%🍁%
%🍁% \item You might want to write a function that takes a word and
%🍁%   computes a list of all the words that can be formed by removing one
%🍁%   letter.  These are the ``children'' of the word.
%🍁% \index{recursive definition}
%🍁% \index{definition!recursive}
%🍁%
%🍁% \item Recursively, a word is reducible if any of its children
%🍁% are reducible.  As a base case, you can consider the empty
%🍁% string reducible.
%🍁%
%🍁% \item The wordlist I provided, {\tt words.txt}, doesn't
%🍁% contain single letter words.  So you might want to add
%🍁% ``I'', ``a'', and the empty string.
%🍁%
%🍁% \item To improve the performance of your program, you might want
%🍁% to memoize the words that are known to be reducible.
%🍁%
%🍁% \end{enumerate}
%🍁%
%🍁% Solution: \url{http://thinkpython2.com/code/reducible.py}.

编写一个程序, 按照这种规则找到所有可以缩减的单词, 然后看看其中哪个词最长。

这道题比大部分的习题都要难, 所以我给出一些建议:

\begin{enumerate}
\item 可能你需要写一个函数将输入单词的所有``子词''(即拿掉一个字母后所有可能的新词)以列表形式输出。
\index{recursive definition} \index{definition!recursive}

\item 递归地看, 如果单词的子词之一也可缩减, 那么这个单词也可被缩减。
我们可以将空字符串视作也可以缩减, 视其为基础情形。

\item 我们提供的词汇表 {\em (\li{words.txt})} 并未包含诸如 {\em `I'}、 {\em `a'} 这样的单个字母词汇, 因此你可能需要加上它们。

\item 为了提高你程序的性能, 你可能需要暂存 {\em (memorize)} 好已被发现的可被缩词的词汇。

\end{enumerate}

\href{http://thinkpython2.com/code/reducible.py}{参考答案}

\end{exercise}




%\begin{exercise}
%\url{http://en.wikipedia.org/wiki/Word_Ladder}
%\end{exercise}



